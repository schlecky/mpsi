\documentclass[a4paper]{tp}
\usepackage[version=3]{mhchem}
\usepackage{tikz}
\usetikzlibrary{decorations.markings}
\usetikzlibrary{arrows.meta}
\titre{TP1 : Lois de Descartes}
\begin{document}
\section{Objectif du TP}

On dispose de matériel spécialement destiné à étudier les lois de la réflexion et de la réfraction : une lampe muni d'une fente envoie un faisceau incident sur un demi-disque en plexiglas. Celui-ci est fixé sur un rapporteur. Si besoin, ajuster la position de la lampe pour que le faisceau passe précisément par le centre du demi-disque.

L'objectif du TP est de vérifier expérimentalement les lois de Snell-Descartes concernant la réflexion et la réfraction. On cherchera notamment à déterminer la relation entre les angles d'incidence et de réflexion.

On attachera une attention particulière à essayer d'estimer les incertitudes associées aux mesures et à déterminer les incertitudes sur les résultats calculés (notamment sur l'indice de réfraction du plexiglas).


\section{\'Etude de la réflexion et de la réfraction sur un dioptre  air-plexiglas} 

On souhaite étudier la réflexion et la réfraction sur le dioptre plan. Pour cela, on fera arriver le pinceau lumineux sur la partie plate du demi-disque, comme sur la figure ci-dessous.

\begin{center}
\begin{tikzpicture}[]
 \draw (0,0) circle(2);
 \draw[fill=gray!20] (0,-1.5) -- (0,1.5) arc (90:-90:1.5);
 \foreach \a in {10,20,...,80} {
   \draw (\a:1.8) -- (\a:2) node[pos=1.1, right, sloped]{\scriptsize \a};
   \draw (-\a:1.8) -- (-\a:2) node[pos=1.1, right, sloped]{\scriptsize \a};
   \draw (180-\a:1.8) -- (180-\a:2) node[pos=1.1, left, sloped]{\scriptsize \a};
   \draw (180+\a:1.8) -- (180+\a:2) node[pos=1.1, left, sloped]{\scriptsize \a};
 }
 \draw (0:1.8) -- (0:2) node[pos=1.1, right, sloped]{\scriptsize 0};
 \draw (180:1.8) -- (180:2) node[pos=1.1, left, sloped]{\scriptsize 0};
 \draw (90:1.8) -- (90:2) node[pos=1.1, right, sloped]{\scriptsize 90};
 \draw (-90:1.8) -- (-90:2) node[pos=1.1, right, sloped]{\scriptsize 90};

 \draw[thick, rayon] (215:3) -- (0,0);
 \draw[thick, rayon] (0,0) -- (25:3);
 \draw[dashed] (-1, 0) -- (1, 0);
 \draw[latex-latex] (60:2.5) arc (60:120:2.5);
 \draw (215:1) arc (215:180:1) node[midway, left]{$i_1$ };
 \draw (0:1) arc (0:25:1) node[midway, right]{$i_2$ };
\end{tikzpicture}
\captionof{figure}{
 Schéma du montage d'étude des lois de Descartes.
}

\end{center}
\begin{itemize}
  \item  Quel est l'intérêt d'avoir un second dioptre de forme circulaire ? 

  \item Vérifier la loi de Snell-Descartes de la réflexion pour plusieurs valeurs de l'angle d'incidence.

  \item Tracer la courbe expérimentale $\sin(i_1)$ en fonction de $\sin(i_2)$, où $i_1$ est l'angle d'incidence sur le dioptre plan et $i_2$ l'angle réfracté.

  \item En appelant $y$ l'ordonnée et $x$ l'abscisse, quelle doit-être théoriquement la fonction $y(x)$ ? Tracer la droite passant au plus près des points expérimentaux et en déduire l'indice $n$ du plexiglas. 
\end{itemize}

\section{Réflexion totale}


\begin{itemize}
  \item Une réflexion totale est-elle possible sur un dioptre air-plexiglas ? Sur un dioptre plexiglas-air ? En déduire comment observer la réflexion totale avec le matériel disponible.
  \item En déduire une autre façon de mesurer l'indice $n$ du plexiglas. 
\end{itemize}

\end{document}

\section{Étude d'un prisme}

On remplace le demi-disque par un prisme en plexiglas. Le faisceau lumineux est envoyé 
sur un côté du prisme, il est réfracté par un
premier dioptre en arrivant sur le prisme puis par un deuxième dioptre en
sortant du prisme. 

\begin{center}
\tikzset{rayon/.style = {postaction={decorate},decoration = {markings, mark = at position 0.5 with {\arrow{latex}}}}}
\begin{tikzpicture}[scale=1.2]
%prisme
\draw[thick] (0,0) -- (-120:3) coordinate (A) node[midway](I){};
\draw[thick] (0,0) -- (-60:3) coordinate (B) node[pos=0.55](R){};
\draw[thick] (A) -- (B);
\draw (-120:0.5) arc (-120:-60:0.5) (0,-0.5) node[below]{A};

%normal rayon incident
\draw[dashed] (I) ++(-210:1) -- ++(-30:2.5);
%rayon incident
\draw[rayon] (I) ++(190:1.5) -- (I.center);
\draw[dashed] (I) -- ++(10:3);
%angle i
\draw (I) ++(150:0.4) node[left,yshift=-0.1cm]{$i$} arc(150:190:0.4);

%rayon intérieur
\draw[rayon] (I.center) -- (R.center);
%angle r
\draw (I)++(-30:0.4) node[right,yshift=1]{$r$} arc(-30:-7:0.4);
%angle r'
\draw (R)++(210:0.4) node[above left,yshift=-4]{$r'$} arc(210:173:0.4);
%normal rayon émergent
\draw[dashed] (R) ++(-150:1)  -- ++(30:1.5);
%rayon émergent
\draw[rayon] (R.center) -- ++(-15:1.5);
%angle i'
\draw (R)++(-15:0.4) node[above right]{$i'$} arc(-15:30:0.4);
%angle D
\draw (R)++(-15:1) arc(-15:10:2) node[below right,yshift=-5]{$D$};
\end{tikzpicture}
\end{center}

\subsection{Détermination de l'angle}

Pour déterminer expérimentalement l'angle $A$, on envoie un faisceau de lumière sur cet angle et on mesure l'angle entre les deux faisceaux réfléchis. 
\begin{center}
  \begin{tikzpicture}
    \draw[thick] (0,0) -- (-60:2) -- (-120:2) -- cycle;
    \draw (-60:0.5) arc (-60:-120:0.5) node[midway, below]{$A$ };
    \coordinate (A) at (-120:0.4);
    \coordinate (B) at (-60:0.4);
    \draw[rayon] ($(A)+(0,1)$) -- (A) ;  
    \draw[rayon] ($(B)+(0,1)$) -- (B);  
    \draw[rayon] (A) -- ++(-150:2);
    \draw[rayon] (B) -- ++(-30:2);
    \draw ($(A)+(-150:1)$) to[out=-60, in=-120] ($(B)+(-30:1)$) ;
    \draw (0, -1.3) node[below] {$2A$} ;
  \end{tikzpicture}
\end{center}

\begin{itemize}
  \item Montrer que l'angle entre les deux faisceaux réfléchis vaut $2A$. On peut le montrer dans le cas où les angles d'incidence sur les deux faces sont les mêmes, mais aussi que ça reste vrai si le prisme est un peu \emph{tourné}.  
  \item Mesurer par cette méthode la valeur de $A$. 
\end{itemize}


\subsection{\'Etude de la déviation}

On s'intéresse à la déviation du prisme c'est à dire l'angle
$D$ entre le rayon incident et le rayon émergent (\cf figure en
annexe). 
\begin{itemize}
  \item Tracer sur le même graphe les courbes expérimentales $D$ en fonction
de $i$ correspondant aux deux configurations.
Constater que les courbes présentent un minimum.
\end{itemize}

\bigskip

On peut montrer que l'indice $n$ du prisme vérifie
$$
n=\frac{\sin\left(\frac{A+D_m}{2}\right)}{\sin\left(\frac{A}{2}\right)},
$$
où $A$ est l'angle du prisme et $D_m$ la valeur minimale de D.

\begin{itemize}
  \item Utiliser cette relation pour déterminer la valeur de l'indice du prisme à
partir des courbes expérimentales.
\end{itemize}


\subsection{Dispersion}

L'indice $n$ dépend de la longueur d'onde donc la déviation dépend
de la longueur d'onde. 

\begin{itemize}
  \item Quelle couleur est la plus déviée par le prisme ?
  \item Peut-on déterminer expérimentalement un ordre de grandeur de
$\Delta n$, où $\Delta n$ est la différence d'indice entre les deux extrémités
du spectre visible ?
\end{itemize}
\end{document}


