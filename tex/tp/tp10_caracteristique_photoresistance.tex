\documentclass[]{tp}
\usepackage{circuitikz}

\titre{TP10 : caractéristique d'une photorésistance}
\begin{document}
%\small

\section{Objectif du TP}
L'objectif de ce TP est de mesurer la caractéristique statique d'une photorésistance. Il s'agit d'un composant électronique dont la résistance électrique dépend de l'éclairement.

\vspace{1em}
\textit{Ne pas oublier qu'une mesure physique doit toujours être associée à une incertitude expérimentale. Penser à lire la notice des appareils pour connaître l'incertitude liée aux valeurs qu'ils fournissent.}

\section{Caractéristique de la photorésistance}

En vous aidant éventuellement du TP précédent, mesurer la caractéristique de la photorésistance et déterminer l'évolution de cette caractéristique avec l'éclairement.


\end{document}
