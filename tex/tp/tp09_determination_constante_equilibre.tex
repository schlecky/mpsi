%! TeX program = lualatex
\documentclass{tp}
\usepackage{chemfig}
\usepackage{mhchem}
\usepackage{makecell}
\usepackage{pgfplots}
\titre{TP9 : Mesure de constantes d'équilibre}
\begin{document}
%\small


\section{Objectif du TP}
L'objectif de ce TP est de mesurer la constante d'équilibre de réactions chimiques.

\section{Dissolution du sulfate de calcium}
\subsection{Étude préliminaire}%
\label{sub:etude_preliminaire}

 Le sulfate de calcium (\ce{CaSO4}), plus connu sous le nom de gypse joue un rôle important dans la résistance mécanique des matériaux à base de ciment.

 Le sulfate de calcium est un solide ionique, lorsqu'il est mis en solution aqueuse, il se dissout selon l'équation :
 \[
 \ce{CaSO4(s) <=> Ca^2+(aq) + SO4^2-(aq) }
 \]
 de constante d'équilibre 
 \[
 K_S = \frac{1}{{c^\circ}^2}[\ce{Ca^2+}][\ce{SO4^2-}]
 \] 
 Lorsque cette réaction atteint un état d'équilibre, c'est-à-dire qu'il reste du \ce{CaSO4(s)} en fin de réaction, on dit que la solution est \emph{saturée}. Si on introduit du \ce{CaSO4(s)} dans de l'eau pure, on a le tableau d'avancement :
 \begin{center}
   \begin{tabular}{@{}llll@{}}
   \toprule
    & \ce{CaSO4(s)} & \ce{Ca^2+(aq)} & \ce{SO4^2-(aq)} \\
    \midrule
    État initial & $n$ & $0$ & $0$ \\
    Équilibre & $n-\xi$ & $\xi$ & $\xi$ \\
    \bottomrule
   \end{tabular}
 \end{center}
 Donc la quantité de \ce{CaSO4(s)} dissoute à l'équilibre dans un volume $V$ de solution est donnée par 
\[
\xi = c^\circ V\sqrt{K_S}
\]
La \emph{solubilité molaire} $S$ du sulfate de calcium est 
\[
 S = \frac{\xi}{V} = c^\circ \sqrt{K_S} \quad \text{en}\quad \si{\mol\per\litre}
\]
Et sa \emph{solubilité massique} est :
\[
s = S\times M(\ce{SO4^2-}) \quad \text{en} \quad \si{\gram\per\litre}
\]

Pour déterminer $\xi$, nous allons mesurer la conductivité électrique de la solution. La conductivité $\sigma$ d'une solution aqueuse contenant des ions \ce{Ca^2+} et \ce{SO4^2-} est :
\[
\sigma = \lambda_{\ce{Ca^2+}}[\ce{Ca^2+}] + \lambda_{\ce{SO4^2-}}[\ce{SO4^2-}] = \frac{\xi}{V}\left( \lambda_{\ce{Ca^2+}} + \lambda_{\ce{SO4^2-}}\right)  = S\left( \lambda_{\ce{Ca^2+}} + \lambda_{\ce{SO4^2-}}\right)
\]
où $\lambda_{\ce{Ca^2+}} = \SI{11.90}{\milli\siemens\meter\squared\per\mol}$ et $\lambda_{\ce{SO4^2-}} = \SI{16.00}{\milli\siemens\meter\squared\per\mol}$ sont les \emph{conductivités ioniques molaires} des deux espèces. 

\subsection{Protocole expérimental}%
\label{sub:protocole_experimental}
\begin{itemize}
  \item Étalonner le plus précisément possible le conductimètre à l'aide d'une solution de chlorure de potassium (\ce{K+ + Cl-}).

  \item Préparer un bécher contenant environ \SI{100}{\milli\litre} d'eau distillée et mesurer sa conductivité.

  \item Transvaser dans un erlenmeyer, ajouter une spatule de sulfate de calcium et mettre la solution à agiter 5 à 10~minutes (vous pouvez profiter de ce temps pour étalonner le pH-mètre). Il reste des cristaux de sulfate de calcium à l'issue de cette agitation : on peut considérer que la solution est saturée.

  \item Arrêter l'agitation, on laissera décanter le système afin de faciliter la filtration

  \item À l'aide de papier filtre, de l'entonnoir sur son support et de l'erlenmeyer, filtrer en versant délicatement le contenu du premier erlenmeyer.

  \item Mesurer la conductivité de la solution obtenue.

  \item Recommencer la même expérience avec deux spatules de \ce{CaSO4}.
\end{itemize}

\subsection{Exploitation des résultats}%
\label{sub:exploitation_des_resultats}
\begin{itemize}
\item Comparer et commenter les résultats des mesures avec une ou deux spatules de \ce{CaSO4}.

\item On mesure la conductivité de l'eau distillée pour prendre en compte la présence d'ions conducteurs dans celle-ci. Comment modifier l'éxpression de $\sigma$ obtenue dans la partie précédente pour les prendre en compte ?

\item Déterminer la valeur de $K_S$ et la comparer à la valeur tabulée à \SI{25}{\celsius} $K_S=\num{4.93e-5}$. 
\end{itemize}
\section{Mesure de la constante d'acidité de l'acide éthanoïque}%
\label{sec:mesure_de_la_constante_d_acidite_de_l_acide_ethanoique}

\subsection{Étude préliminaire}%
\label{sub:etude_preliminaire}
L'acide éthanoïque est un acide faible de formule \ce{CH3COOH}. Dans l'eau il se dissocie partiellement en ion éthanoate \ce{CH3COO-} et ion \ce{H+}, selon l'équation de réaction :
\[
\ce{CH3COOH(aq) + H2O(\ell) <=> CH3COO-(aq) + H3O+(aq)}
\]

La constante de cet équilibre est la constante d'acidité $K_a$ de l'acide éthanoïque.

Lorsqu'on introduit une quantité $n$ d'acide éthanoïque dans l'eau, on peut construire le tableau d'avancement :
\begin{center}
  \begin{tabular}{@{}llll@{}}
  \toprule
   & \ce{CH3COOH} & \ce{CH3COO-} & \ce{H3O+} \\
   \midrule
   État initial & $n$ & 0 & 0 \\
   Équilibre & $n-\xi$  & $\xi$ & $\xi$ \\
   \bottomrule
  \end{tabular}
\end{center}

La constante d'équilibre de cette réaction est 
\[
K_a = \frac{(\xi/V)^2}{(n-\xi)/V\times c^\circ}
\]
En notant $x=\xi/V$ et $c=n/V$ on obtient : 
\[
K_a = \frac{1}{c^\circ}\frac{x^2}{c-x}
\]

Si on néglige les ions $\ce{H3O+}$ apportés par l'autoprotolyse de l'eau, on rappelle que le pH de la solution est défini comme $\mathrm{pH} = -\log([\ce{H3O+}]/c^\circ)$ et on obtient la relation 
\[
K_a  = \frac{10^{-2\,\mathrm{pH}}}{c/c^\circ-10^{-\mathrm{pH}}}
\]

\begin{itemize}
  \item En écrivant l'équilibre d'autoprotolyse de l'eau, dont la constante $K_e$ vaut \num{e-14}, déterminer la concentration en ions \ce{H3O+} dans l'eau pure et le pH de cette dernière. Proposer une valeur du pH à partir de laquelle on pourra effectivement négliger les ions \ce{H3O+} apportés par l'autoprotolyse de l'eau par rapport à ceux apportés par l'acide éthanoïque. 
\end{itemize}

\subsection{Le pH-mètre}%
\label{sub:phmetre}
Avant de mesurer le pH, il faut étalonner le pH-mètre. Un pH-mètre mesure une différence de potentiel entre une référence et une electrode \emph{de verre} plongée dans la solution étudiée et dont le potentiel est sensible à la concentration en ions \ce{H3O+}. La relation entre cette différence de potentiel et le pH est affine. 
Étalonner le pH-mètre, c'est ajuster la pente et l'ordonnée à l'origine de cette relation à partir d'un solution de pH connu.

Généralement on commence par étalonner avec une solution de pH égal à 7 puis, si on travaille en milieu acide, avec une solution de pH égal à 4. Si on travaille en milieu basique on utilisera plutôt une solution basique de pH égal à 9. Le protocole d'étalonnage peut varier selon les modèles. Reportez-vous à la notice du pH-mètre dont vous disposez.

Les électrodes ne doivent jamais rester à l'air libre, et en raison de sa constitution, l'electrode de verre ne doit pas effectuer de séjour prolongé dans une solution basique.

Avant chaque plongée d'une électrode dans une nouvelle solution, la ricer à l'eau distillée au-dessus d'un bécher et l'essuyer au papier Joseph.

\subsection{Protocole expérimental}%
\label{sub:protocole_experimental}

\begin{itemize}
  \item Étalonner le pH-mètre.

  \item Prélever \SI{50}{\milli\litre} d'acide éthanoïque à \SI{e-2}{\mol\per\litre}

  \item En utilisant la fiole et la pipette jaugée, préparer successivement \SI{100}{\milli\litre} de solution de concentration \SI{e-3}{\mol\per\litre}, \SI{e-4}{\mol\per\litre}, \SI{e-5}{\mol\per\litre} et \SI{e-6}{\mol\per\litre}.

  \item Mesurer le pH de chacune des 5 solutions. Ne pas les jeter.
\end{itemize}

\subsection{Exploitation des résultats}%
\begin{itemize}
  \item Utiliser les différentes valeurs de pH mesurées pour estimer la valeur de $K_a$. Et la comparer à la valeur tabulée $K_a = 10^{\num{-4.76}}$. 

\end{itemize}
\label{sub:exploitation_des_resultats}

\subsection{Mesure par conductimétrie}%
\label{sub:mesure_par_conductimetrie}

\begin{itemize}
  \item Proposer et mettre en œuvre un protocole permettant de mesurer la constante $K_a$ par conductimétrie.

  On donne $\lambda_{\ce{H3O+}}=\SI{35.0}{\milli\siemens\meter\squared\per\mol}$ et $\lambda_{\ce{CH3COO-}}=\SI{4.09}{\milli\siemens\meter\squared\per\mol}$
  \item Comparer à la valeur obtenue dans la partie précédente.
\end{itemize}




\end{document}
