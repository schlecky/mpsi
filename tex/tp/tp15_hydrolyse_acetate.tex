\documentclass{tp}
\usepackage{chemfig}
\usepackage{mhchem}
\usepackage{makecell}
\usepackage{pgfplots}
\titre{TP16 :Suivi de l'hydrolyse de l'acétate d'éthyle par conductimétrie}

\begin{document}
\noindent \textsc{Objectifs du TP}
\begin{itemize}
\item Utiliser un appareil de mesure (conductimètre) en s'aidant d'une notice.

\item Suivi cinétique d'une transformation chimique : mettre en \oe uvre une méthode de suivi temporel, et exploiter les résultats pour déterminer les caractéristiques cinétiques d'une réaction.
\end{itemize}


\section{Généralités sur la conductimétrie}

\subsection{Introduction}

\indent La conductimétrie est une méthode physique utilisée fréquemment en chimie. Elle repose sur les propriétés électriques des solutions ioniques :

\begin{itemize}
\item dans une solution, seuls les ions peuvent transporter le courant électrique
\item les contributions de tous les ions s'ajoutent
\end{itemize}


\indent La conductivité se mesure en faisant passer un courant alternatif dans la solution. En effet en courant continu, soit il y aurait conduction et électrolyse, soit après un régime transitoire le courant deviendrait nul et on ne pourrait rien mesurer (condensateur en régime continu).


\subsection{Schéma électrique équivalent}

\indent Un conductimètre est composé de deux plaques parallèles de surface~$S$, distante de~$\ell$. Ces deux plaques forment la \textbf{cellule} de conductimétrie. \'Electriquement, la portion de solution comprise entre ces deux plaques est équivalente à une résistance.

\vspace{-0.4cm}
%\begin{figure}[!h]
%\begin{center}
%\centrefigure{PS/tp_conductimetrie}
%\end{center}
%\end{figure}\\

\indent Le conductimètre mesure la conductance (l'inverse de la résistance) de la portion de solution contenue dans la cellule.

\subsection{Relation entre la conductance et la conductivité}

\indent Vous démontrerez en deuxième année\footnote{Mais cette relation est à retenir tout de suite !} que la résistance d'un cylindre conducteur homogène de longueur~$\ell$ et de section~$S$ vaut $R = \frac{\rho \ell}{S}$ où $\rho$ est la \textbf{résistivité} du matériau en \si{\ohm\meter}.

\indent Cette relation est aussi valable pour une portion parallélépipédique d'une solution ionique : $G = \frac{\sigma S}{\ell}$, où $\sigma=\frac{1}{\rho}$ est la conductivité de la solution en $\si{\siemens\per\meter}$, $G$ la conductance de la portion de solution en S, $S$ la surface d'une plaque de la cellule en $\si{\meter^2}$ et $\ell$ la longueur de la cellule en \si{\meter}.

\subsection{La constante de cellule}

\indent On cherche à séparer les natures des contributions à la valeur mesurée en écrivant la conductance $G$ sous la forme :

\vspace{-0.5cm}

\centers{$G = \text{caractéristique électrique} \times \text{caractéristiques géométriques}$}

\indent Les caractéristiques géométriques sont données par $S$ et $\ell$. On choisit donc de définir la \textbf{constante de cellule} :
\centers{$\boxed{K\e{cell} = \f{\ell}{S}}$ \qquad \text{de manière à ce que} \qquad $G = \sigma \times \f{1}{K\e{cell}}$}

\indent Pour les conductimètres utilisés en TP, cette constante est de l'ordre du$\U{cm^{-1}}$. On pourrait la déterminer en mesurant la cellule mais cette méthode est peu précise. C'est la raison pour laquelle on préfère utiliser une solution étalon (de conductivité connue).


\subsection{Conductivité molaire limite}

\subsubsection{Définition}

\indent En définissant $\sigma = K\e{cell} \, G$, on s'est affranchi des caractéristiques liées à la géométrie de la cellule. Il est pratique de
se ramener à la \textbf{conductivité molaire} $\lambda$ en$\U{S.m^2.mol^{-1}}$ telle que :
\centers{$\boxed{\sigma = \lambda \, c}$}

\noindent où $c$ est la concentration de la solution en l'ion considéré.

\medskip

\indent Ainsi pour une solution d'acide chlorhydrique : $\sigma = \lambda_{(H^+)} \, [H^+] + \lambda_{(\Cl^-)} \, [\Cl^-]$, puisque comme on l'a dit en introduction, les conductivités des ions s'ajoutent.

\medskip

\indent Dans le cas général, la conductivité molaire dépend de la concentration. Elle est cependant constante pour des solutions diluées\footnote{Aux
concentrations élevées, la conductivité n'est plus proportionnelle à la concentration.}. C'est la raison pour laquelle on trouve dans les tables la \textbf{conductivité molaire limite} (pour des solutions infiniment diluées) : on la note $\lambda^0_{(\text{X}_i)}$, pour l'ion X$_i$.

\medskip

\indent Pour une solution diluée d'acide chlorhydrique, on écrira : $\sigma = \lambda^0_{(H^+)} \, [H^+] + \lambda^0_{(\Cl^-)} \, [\Cl^-]$.

\vspace{-0.4cm}

\subsubsection{Conductivité molaire limite équivalente}

\indent Certaines tables donnent, non pas la conductivité molaire limite, mais la \textbf{conductivité molaire limite équivalente}, qui correspond à la conductivité molaire limite pour une unité de charge de l'ion. Pour les ions monovalents, cela ne change rien. En revanche, il faut se méfier dans les autres cas.

\indent Ainsi pour une solution de sulfate de calcium : $\sigma = \lambda^0_{(\Ca^{2+})} \, [\Ca^{2+}] + \lambda^0_{(SO_4^{2-})} \, [SO_4^{2-}]$ si on utilise la conductivité molaire limite des ions. Mais $\sigma = 2 \, \lambda^0_{(\Ca^{2+})} \, [\Ca^{2+}] + 2 \, \lambda^0_{(SO_4^{2-})} \, [SO_4^{2-}]$ si $\lambda^0$ désigne la conductivité molaire limite équivalente (\ie par unité de charge). Elle est parfois notée $\lambda^0_{(\f{1}{2}\,\Ca^{2+})}$ ou $\lambda^0_{(\f{1}{2}\,SO_4^{2-})}$ pour marquer la différence.

\indent \underline{Prenez donc toujours garde à la conductivité qui est utilisée.}\\

\vspace{-0.4cm}

\section{Partie théorique sur la cinétique}

\subsection{Objectifs expérimentaux}

\begin{itemize}
\item Suivre la cinétique de la saponification de l'acétate d'éthyle (éthanoate d'éthyle en nomenclature officielle) par l'hydroxyde de sodium de constante de vitesse~$k$ :


\centers{$CH_3COOC_2H_5\aq + HO^-\aq \rightarrow CH_3COO^-\aq + C_2H_5OH\aq$}

\item Vérifier que cette réaction est d'ordre global égal à~$2$.
\item Vérifier que chacun des ordres partiels est égal à~$1$.
\item Déterminer la constante de vitesse.
\end{itemize}

\subsection{\'Etude quantitative}

\indent On note $a$ la concentration initiale en ester et en ion hydroxyde, $x$ l'avancement volumique.

\begin{enumerate}

\item $\clubsuit$ Grâce aux hypothèses sur les ordres partiels de la réaction, donner la relation entre $a$, $x$, $t$ et $k$.

\setcounter{save}{\arabic{enumi}}
\end{enumerate}

\indent On rappelle que la conductivité de la solution est : $\boxed{\sigma = \sum_i \lambda_i^0(M_i^{z_i +}) \, . \, c_i}$ avec $z_i$ un entier relatif, $\lambda_i^0(M_i^{z_i +})$ la conductivité molaire limite de l'ion $M_i^{z_i +}$, et $c_i$ sa concentration.

\medskip

\indent On note $\sigma_0$ et $\sigma_{\infty}$ respectivement les~conductivités initiale et finale.

\begin{enumerate}
\setcounter{enumi}{\thesave}

\item $\clubsuit$ Après avoir recensé \textbf{tous} les ions en solution, montrer que l'on peut écrire la relation : 

\encadre{$\f{\sigma_0-\sigma}{\sigma-\sigma_{\infty}} = k\,a\,t$}

\setcounter{save}{\arabic{enumi}}
\end{enumerate}

\indent Un mélange identique à celui que vous allez étudier a été préparé 24 heures auparavant. Sa conductivité a été mesurée à la température de la salle. On~considère donc $\sigma_{\infty}$ comme connue.

\begin{enumerate}
\setcounter{enumi}{\thesave}

\item $\clubsuit$ En déduire la courbe à tracer pour obtenir le plus simplement possible les valeurs de $k$ et de $\sigma_0$.

\setcounter{save}{\arabic{enumi}}
\end{enumerate}

\section{Partie expérimentale}

Elle sera réalisée à l'aide du conductimètre que l'on n'oubliera pas d'étalonner. Une notice est à votre disposition sur votre paillasse. Les informations essentielles figurent en plus à la fin de cet énoncé.

\subsection{Préliminaires}

\begin{itemize}
\item [\textbullet] Préparer la série de béchers suivante :
  \begin{itemize}
  \item	Bécher $1$ : environ $20 \U{mL}$ de KCl à $10^{-1} \U{mol.L^{-1}}$, afin d'effectuer l'étalonnage du conductimètre (seulement si votre conductimètre nécessite un étalonnage)
  \item	Bécher $2$ : $20 \U{mL}$ d'acétate d'éthyle à $5.10^{-2} \U{mol.L^{-1}}$
  \item	Bécher $3$ : $20 \U{mL}$ d'hydroxyde de sodium à $5.10^{-2} \U{mol.L^{-1}}$
  \item	Bécher $4$ : $20 \U{mL}$ d'hydroxyde de sodium à $5.10^{-2} \U{mol.L^{-1}}$ auquel on ajoute $20 \U{mL}$ d'eau distillée
  \end{itemize}
\end{itemize}
	
\subsection{Mesure}

On utilisera Latispro$\circledR$ en mode \ofg{pas à pas}.

\subsubsection{Mesure de la conductivité de la solution du bécher $4$}

\begin{itemize}
\item [\textbullet] Ouvrir le tableau de bord (\ofg{Outils}, \ofg{Tableau de bord}).

\item [\textbullet] Faire la mesure et la noter.
\end{itemize}

\subsubsection{Mesure au cours du temps}

\begin{itemize}
%\item[\textbullet] Lancer le logiciel \ofg{Acqui Poinca}.
%\item[\textbullet] Sélectionner l'appareil de mesure (conductimètre).
%\item[\textbullet] Se placer en mode cinétique et choisir une mesure toutes les $30 \U{s}$.
\item[\textbullet] Mettre le bécher $2$ sur l'agitateur magnétique avec un barreau aimanté, rincer et essuyer l'extérieur de la sonde et
l'introduire dans le bécher (\textbf{On fera attention à ce que le barreau ne puisse pas cogner contre la cellule}).

\item [$\bullet$] Appuyer sur la flèche bleu pour avoir accès à la fenêtre \ofg{Acquisition pas à pas}.
\item[$\bullet$] Verser la solution d'hydroxyde de sodium (bécher $3$) dans le bécher $2$ et déclencher simultanément le chronomètre. Vérifier que la cellule trempe complètement dans le bécher sans que le barreau puisse cogner dessus, agiter une dizaine de secondes.
\item[$\bullet$] Stopper l'agitation

%et lancer la mesure, simultanément arrêter le chronomètre.

\item [$\bullet$] Toutes les 30s et pendant 15 minutes, relever manuellement la conductivité.
\item[$\bullet$] Relever la température du milieu réactionnel.
\item[$\bullet$] Garder le milieu réactionnel afin de le laisser évoluer.
\item[$\bullet$] Mesurer à la fin du TP, la valeur de $\sigma$.
\end{itemize}

%\begin{itemize}
%\item[$\bullet$] Lancer le logiciel $G(t)$ qui doit se trouver dans le répertoire \ofg{Acquichimie" présent sur le bureau.
%\item[$\bullet$] Paramètres d'acquisition du logiciel à valider : une mesure toutes les $30 \,$s pendant $15\,$min.
%\item[$\bullet$] Mettre le bécher $2$ sur l'agitateur magnétique avec un barreau aimanté, rincer et essuyer l'extérieur de la sonde et
%l'introduire dans le bécher (\textbf{On fera attention à ce que le barreau ne puisse pas cogner contre la cellule}).
%\item[$\bullet$] Verser la solution d'hydroxyde de sodium (bécher $3$) dans le bécher $2$ et déclencher simultanément le chronomètre. Vérifier que la cellule trempe complètement dans le bécher sans que le barreau puisse cogner dessus, agiter une dizaine de secondes.
%\item[$\bullet$] Stopper l'agitation et lancer la mesure en cliquant sur \ofg{commencer", simultanément arrêter le chronomètre.
%\item[$\bullet$] Relever la température du milieu réactionnel.
%\item[$\bullet$] Garder le milieu réactionnel afin de le laisser évoluer.
%\item[$\bullet$] Mesurer à la fin du TP, la valeur de $\sigma$.
%\end{itemize}


\subsubsection{Exploitation}


\begin{enumerate}
\setcounter{enumi}{\thesave}

\item Dresser un tableau regroupant les différentes conductivités mesurées.
\item Comparer la valeur obtenue en fin de TP à celle de $\sigma_{\infty}$ obtenue au bout de $24 \U{h}$.
\item Comparer le $\sigma_0$ obtenu à la conductivité du bécher $4$.
\item En utilisant soit Regressi$\circledR$, soit Excel$\circledR$, valider l'ordre $2$, déterminer la constante de vitesse et $\sigma_0$.

\setcounter{save}{\arabic{enumi}}
\end{enumerate}


\`A titre indicatif, on donne :
\begin{table}[!h]
\begin{center}
\begin{tabular}{|c|c|c|c|c|c|c|c|c|}
\hline
ion & $H_3O^+$ & $\Na^+$ & $K^+$ & $\Pb^{2+}$  & $\Cl^-$ & $HO^-$ & $CH_3COO^-$ & $SO_4^{2-}$\\
\hline
$\lambda^0 \U{(mS.m^2.mol^{-1})}$ & 35,0 & 5,01 & 7,35 & 14,0 & 7,63 & 19,9 & 4,09 & 16 \\
\hline
\end{tabular}
\end{center}
\end{table}

\vspace{-1cm}

\section{\`A propos de Latispro$\circledR$}

\begin{itemize}

\item [\textbullet] Le logiciel Latispro$\circledR$ s'ouvre par \ofg{Logiciels Physique-Chimie} (sur le bureau), puis en passant par le dossier \ofg{Chimie} et non \ofg{Physique}.

\item [\textbullet] Si la fenêtre d'étalonnage ne s'affiche pas au démarrage de Latispro$\circledR$, aller dans \ofg{Edition}, \ofg{Capeurs d'utilisateurs} et choisir le bon capteur. Si ce dernier n'apparaît pas, vérifier la bonne connexion de la cellule conductimétrique à la console de Latispro$\circledR$. Redémarrer l'ordinateur et réitérer la démarche.

\end{itemize}

\end{document}



