\documentclass{tp}
\usepackage{pgfplots}
\titre{TP12 : Charge d'un condensateur}
\begin{document}
%\small

\section{Objectif du TP}
L'objectif de ce TP est de faire l'acquisition de la tension aux bornes du condensateur au cours de sa charge à travers une résistance, et de confronter les résultats expérimentaux aux expressions théoriques.

\vspace{1em}
\textit{Ne pas oublier qu'une mesure physique doit toujours être associée à une incertitude expérimentale. Penser à lire la notice des appareils pour connaître l'incertitude liée aux valeurs qu'ils fournissent.}

\section{Acquisition de la tension aux bornes du condensateur}
\subsection{Visualisation à l'oscilloscope}
Réaliser un montage permettant de visualiser la charge du condensateur à l'oscilloscope.

Déterminer par cette méthode la constante de temps $\tau$ de la charge et la confronter à la valeur théorique qui est : $\tau=RC$. Ne pas oublier d'évaluer l'incertitude associée à cette mesure.

On rappelle que la valeur de la constante de temps associée à la charge du condensateur peut être déterminée grâce à la méthode de la tangente représentée sur la figure ci-dessous.
\begin{center}
%includegraphics[width=0.4\linewidth]{TP10_charge_cond.pdf}
\begin{tikzpicture}
\begin{axis}[
  height=6cm,
  width=10cm,
  xmin=0,xmax=10,
  ymin=0,ymax=10,
  xtick=\empty,
  ytick=\empty,
  axis lines=left,
  clip=false,
  xlabel=$t$,
  ylabel=$u(t)$,
  every axis y label/.style={at={(axis description cs:0,1)},anchor=south},
  every axis x label/.style={at={(axis description cs:1,0)},anchor=west},
  ]
\addplot[domain=0:10,samples=100,smooth,thick]{8*(1-exp(-x/2))};
\draw[dotted] (axis cs:0,8) node[left] {$E$}-- (axis cs:10,8);
\draw[dotted] (axis cs:2,0) node[below] {$\tau$}-- (axis cs:2,10);
\draw[dashed] (axis cs:0,0) -- (axis cs:2,8);

\end{axis}
\end{tikzpicture}
\end{center}


\subsection{Acquisition informatique} 
\`A l'aide de l'interface d'acquisition et du logiciel \emph{LatisPro}, faire la même acquisition de la charge du condensateur.

L'acquisition informatique permet de confronter directement l'évolution temporelle de la tension aux bornes du condensateur avec l'évolution prévue théoriquement. 

On rappelle que la théorie prévoit que la tension $u(t)$ aux bornes d'un condensateur chargé sous une tension $E$ à travers une résistance $R$ est :
\begin{equation}
	u(t)=E\left(1-\exp\left(-\frac t\tau\right)\right)
\end{equation}

Utiliser la fonction modélisation du logiciel \emph{LatisPro} pour modéliser l'évolution mesurée expérimentalement par la courbe théorique en laissant le logiciel ajuster les paramètres $E$ et $\tau$. Confronter les valeurs obtenues avec les valeurs attendues.
\end{document}
