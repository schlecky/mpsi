\documentclass[]{tp}
\usepackage{multicol}
%\usepackage[version=3]{mhchem}
\titre{TP30 : Calorimétrie 2}
\begin{document}
%\small

\section{Objectif du TP}
Dans ce TP, nous allons à nouveau mesurer la capacité thermique d'un métal et l'enthalpie massique de fusion de la glace comme au TP28, mais en utilisant une méthode différente.

\vspace{1em}
\textit{Ne pas oublier qu'une mesure physique doit toujours être associée à une incertitude expérimentale. Penser à lire la notice des appareils pour connaître l'incertitude liée aux valeurs qu'ils fournissent.}  

\section{Détermination des capacités thermiques}%
\label{capacite_thermique_calorimetre}
\subsection{Protocole expérimental}%
\label{sub:Protocole expérimental}

On commence par déterminer la capacité thermique $C$ du calorimètre et de ses accessoires ainsi que la capacité thermique massique $c_e$ de l'eau. Pour cela, on procède de la manière suivante :

\begin{itemize}
  \item On place une masse $m_1=\SI{200}{\gram}$ d'eau dans le calorimètre sur lequel on a adapté une résistance chauffante. On vérifiera bien que la résistance est complètement immergée, sinon il faudra rajouter de l'eau (il faut connaitre précisément la masse d'eau).

  \item Lorsque la température dans le calorimètre est stable, on fait passer un courant dans la résistance et on mesure l'évolution de la température $T(t)$. On mesurera l'intensité $i_1$ et la tension $u_1$ de la résistance au cours de l'expérience.

  \item On fait l'acquisition de l'évolution de la température en agitant correctement l'eau pour que la température y soit homogène. La température devrait évoluer de manière affine avec le temps. Lorsqu'on observe une droite suffisamment bien définie pour déterminer sa pente, on arrête l'acquisition.

  \item On répète les étapes précédentes avec une masse $m_2=\SI{300}{\gram}$ d'eau dans le calorimètre. 
\end{itemize}

\subsection{Analyse des résultats}%
\label{sub:Analyse des résultats}
Normalement, la tension $u$ et l'intensité $i$ du courant sont relativement constants au cours de l'expérience. La puissance thermique fournie par la résistance est $P=\dt{Q}=ui$. Le premier principe appliqué à l'eau et au calorimètre est :
\begin{equation}
  \Delta H = Q = (m_ec_e + C)\Delta T \quad \text{donc} \quad P = \dt{Q} = (m_ec_e + C) \dt{T}
\end{equation}
En notant $\alpha_1$ la valeur de $\dt{T}$ (pente de la droite) lors de la première expérience, et $\alpha_2$ sa valeur lors de la seconde expérience, on peut montrer (faites-le pour vous entraîner) que :
\begin{equation}
c_e = \frac{\alpha_2 P_1 - \alpha_1P_2}{\alpha_1\alpha_2(m_1-m_2)} \quad \text{et} \quad C = \frac{\alpha_2 m_2 P_1 - \alpha_1m_1P_2}{\alpha_1\alpha_2(m_1-m_2)}
\end{equation}
où $P_1$ et $P_2$ sont les puissances mesurées au cours des deux expériences.

\section{Détermination de l'enthalpie de fusion de la glace}

On souhaite mesurer l'enthalpie de fusion de la glace avec une méthode électrique. Le protocole détaillé est laissé à votre appréciation, mais on pourra éventuellement :
\begin{itemize}
  \item Mettre de l'eau à \SI{0}{\celsius} dans le calorimètre ;
  \item Ajouter une masse $m_g$ de glace connue ;
  \item Faire passer un courant dans la résistance et mesurer le temps que la glace met pour fondre totalement.
  \item Lorsque toute la glace a fondu, éteindre l'alimentation électrique et mesurer l'évolution de la température de l'eau pour estimer le transfert thermique avec l'extérieur.
\end{itemize}
À vous de régler les détails, ou même de modifier le protocole.

\end{document}
