\documentclass[]{tp}
\usepackage{multicol}
\usepackage[version=3]{mhchem}
\titre{TP31 : Cristallographie}
\begin{document}
%\small

\section{Objectif du TP}
Le but de ce TP est d'étudier quelques structures cristallines en s'aidant de modèles numériques. C'est un TP qui ressemble beaucoup à un TD, il ne se prête pas vraiment à la rédaction d'un compte-rendu.

Tous les modèles présentés ici sont interactifs. On pourra faire des mesures de distances en double-cliquant sur un premier atome puis sur un second.  

\section{La structure cubique faces centrées}
Cette structure a été vue en détails en cours. Rendez-vous sur la page : \url{https://mpsi.schleck.ovh/cristaux/cfc.html} 

\begin{center}
\includegraphics[width=5cm]{images/cfc.png}
\end{center}

Assurez-vous que vous pouvez :
\begin{enumerate}
  \item Décrire la maille et sa construction à partir d'un empilement compact de sphères dures.
  \item Déterminer sa population.
  \item Expliquer pourquoi la coordinence d'un atome est de 12.
  \item Situer les sites tétraédriques et octaédriques et visualiser leur position dans l'empilement compact des plans.
\end{enumerate}
Les calculs d'habitabilité des sites intersticiels doivent être connus, mais on ne les refera pas ici.


\section{Le diamant}
La structure cristalline du diamant est présentée sur cette page : \url{https://mpsi.schleck.ovh/cristaux/diamant.html}. Tous les atomes sont des atomes de carbone dont la masse molaire est $M_{\ce{C}} = \SI{12}{\gram\per\mol}$.
\begin{center}
  \includegraphics[width=5cm]{images/diamant.png}
\end{center}
\begin{enumerate}
  \item Observer la maille du diamant pour reconnaitre une structure cubique faces centrées dans laquelle un site tetraédrique sur deux est occupé par un atome de carbone.
  \item Déterminer la population d'une maille et la coordinence des atomes.
  \item Déterminer la masse volumique du diamant.
  \item Déterminer la relation entre le rayon covalent $r$ de l'atome de carbone et le paramètre de maille $a$. En déduire la valeur de $r$. Vérifier en effectuant la mesure sur le modèle.
  \item Déterminer la compacité de la structure.
\end{enumerate}

\section{La structure hexagonale compacte}
La structure hexagonale compacte est la structure cristalline de plusieurs métaux dont le titane, le zinc ou le cobalt. La structure du titane est présentée sur la page \url{https://mpsi.schleck.ovh/cristaux/hc.html}
\begin{center}
  \includegraphics[width=5cm]{images/hc.png}
\end{center}
\begin{enumerate}
  \item Utiliser le modèle pour visualiser comment se construit cette structure à partir d'un empilement de plans compacts.
  \item Repérer les sites intersticiels tétraédriques et octaédriques et vérifier qu'ils s'insèrent entre les plans comme dans la structure CFC.
  \item On peut représenter cette structure par sa maille primitive (la plus petite possible) ou par une maille conventionnelle hexagonale. La maille primitive a une base losange de côté $a$ et une hauteur $c$. Déterminer par le calcul le rapport $c/a$ et vérifier la valeur sur le modèle.  
  \item Déterminer la masse volumique du titane.
\end{enumerate}

\section{Le trioxyde de tungstène}
Le trioxyde de tungstène \ce{WO3} solide est, en première approche, un solide ionique. Il présente une structure cubique représentée sur cette page : \url{https://mpsi.schleck.ovh/cristaux/wo3.html}.
\begin{center}
  \includegraphics[width=5cm]{images/WO3.png}
\end{center}
\begin{enumerate}
  \item Déterminer la compacité du cristal.
  \item Le centre du cube et le centre des faces sont vides. Calculer le rayon maximal d'un élément qui pourrait s'insérer dans ces sites sans déformation. Vérifier le résultat sur le modèle.
\end{enumerate}

\end{document}
