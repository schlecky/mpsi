%! TeX program = lualatex
\documentclass{tp}
\usepackage{pgfplots}
\usepackage{mhchem}
\pgfplotsset{compat=1.17}
\titre{TP26 : Piles et électrodes}
\begin{document}
%\small

\section{Objectif du TP}
L'objectif de ce tp est d'effectuer quelques réactions d'oxydoréduction, et de réaliser des piles pour mesurer les différences de potentiels entre les couples oxydant/réducteur. On étudiera enfin la loi de Nernst.

\section{Réactions d'oxydoréduction}%
\label{sec:reactions_d_oxydoreduction}

\subsection{Réaction entre les ions permanganates et les ions fer (II) en milieu acide}%
\label{sub:reaction_entre_les_ions_permanganates_et_les_ions_fer_ii_en_milieu_acide}

\begin{itemize}
  \item Dans un tube à essai, verser \SI{0.5}{\milli\litre} d'une solution de concentration \SI{0.1}{\mol\per\litre} de permanganate de potassium acidifié (\ce{K+ + MnO4-}). Noter la couleur de la solution. À quelle espèce chimique est due cette couleur ?

  \item Ajouter progressivement une solution de sulfate ferreux (\ce{Fe^2+ + SO4^2-}) à \SI{0.1}{\mol\per\litre}.

  \item Noter vos observations. Quelle réaction a lieu ? Déterminer l'oxydant et le réducteur. Pouvait-on s'attendre à observer cette réaction sachant qu'à \SI{25}{\celsius}, $E^0(\ce{MnO4- / Mn^2+})=\SI{1.51}{\volt}$ et $E^0(\ce{Fe^3+/Fe^2+})=\SI{0.77}{\volt}$ ?  
\end{itemize}

\subsection{Action du fer sur les ions cuivre (II)}%
\label{sub:action_du_fer_sur_les_ions_cuivre_ii_}

\begin{itemize}
  \item Dans un tube à essai, placer un morceau de paille de fer.
  \item Ajouter quelques millilitres de la solution de sulfate de cuivre (\ce{Cu^2+ + SO4^2-)} à \SI{0.1}{\mol\per\litre} et noter les observations.

  \item Au bout d'une minute, prélever environ \SI{2}{\milli\litre} de la solution, les placer dans un tube à essai et y ajouter avec précaution quelques gouttes d'hydroxyde de sodium (\ce{Na+ + HO-}) à \SI{0.1}{\mol\per\litre}. D'après ce test, quels sont les produits de la première réaction. Écrire l'équation de la réaction qui a eu lieu entre le sulfate de cuivre et le fer, puis celle de la réaction avec la soude.

  \item Justifier ces observations, sachant que $E^0(\ce{Cu^2+ / Cu}) = \SI{0.34}{\volt} $ et $E^0(\ce{Fe^2+ / Fe})=\SI{-0.44}{\volt} $  
\end{itemize}

\section{La pile Daniell}
\label{sec:la_pile_daniell}

\subsection{Mode opératoire}
\label{sub:mode_operatoire}

\begin{itemize}
  \item Dans un petit bécher, mettre une solution de sulfate de cuivre à \SI{0.1}{\mol\per\litre} et tremper une lame de cuivre préalablement nettoyée à la toile émeri. C'est la première demi-pile.

  \item Dans le seconde bécher, mettre une solution de sulfate de zinc de même concentration et tremper une lame de zinc également propre. C'est la seconde demi-pile.

  \item Finalement, on ferme le circuit à l'aide d'un pont salin pour former la pile.
\end{itemize}

\begin{center}
  \begin{tikzpicture}
  %tikz redox
    \fill[gray!20] (0,0) rectangle (2,2);
    \draw (0,2.5) -- (0,0) -- node[above]{\footnotesize\ce{Cu^2+}}(2, 0) -- (2, 2.5);
    \draw[fill=gray, rotate around={10:(0.5,1)}] (0.5,1) rectangle (1, 3) (0.75,2) node{\footnotesize Cu} (0.75, 3) coordinate(A);
    \begin{scope}[xshift=4cm]
    \fill[gray!20] (0,0) rectangle (2,2);
    \draw (0,2.5) -- (0,0) -- node[above]{\footnotesize\ce{Zn^2+}}(2, 0) -- (2, 2.5);
    \draw[fill=gray, rotate around={-10:(1,1)}] (1,1) rectangle (1.5, 3) (1.25,2) node{\footnotesize Zn}(1.25, 3) coordinate(B);
    \end{scope}
    \fill[gray!30] (1.5, 1.5) [rounded corners]-- (1.5, 3) -- (4.5, 3) [sharp corners]-- (4.5, 1.5) -- (4.2, 1.5) [rounded corners]-- (4.2, 2.7)-- (1.8, 2.7) --(1.8, 1.5);
    \draw[rounded corners] (1.5, 1.5) -- (1.5, 3) --node[above]{\footnotesize pont salin} (4.5, 3) -- (4.5, 1.5);
    \draw[rounded corners] (1.8, 1.5) -- (1.8, 2.7) -- (4.2, 2.7) -- (4.2, 1.5);
    \coordinate (C) at (0,4);
    \draw (A) -- (A|-C) to[rmeter, t=V, n=M] (B|-C) -- (B);
    \draw (M.left) node[above left]{$+$};
    \draw (M.right) node[above right]{Com};
  \end{tikzpicture}
  \captionof{figure}{Pile Daniell}
\end{center}

\subsection{Expériences}%
\label{sub:experiences}

\begin{itemize}
  \item Fabriquer d'autres piles avec des solutions de nitrate de plomb et de nitrate d'argent, en gardant le cuivre comme première demi-pile. Noter les tensions mesurées.
\end{itemize}

\section{Loi de Nernst}%
\label{sec:loi_de_nernst}

\subsection{Objectif}%
\label{sub:objectif}

On cherche à montrer que le potentiel $E$ du couple \ce{Cu^2+ / Cu} s'écrit sous la forme
\begin{equation}
  E = E^0 + k \ln \left( \frac{[\ce{Cu^2+}]}{\cz} \right) 
\end{equation}

\subsection{Protocole}%
\label{sub:protocole}

\begin{itemize}
  \item Préparer, par dilutions successives, une série de solutions de sulfate de cuivre de concentrations respectives \SI{e-1}{\mol\per\litre}, \SI{e-2}{\mol\per\litre}, \SI{e-3}{\mol\per\litre} et \SI{e-4}{\mol\per\litre}. Ces solutions seront mises dans des béchers de \SI{50}{\milli\litre} en attendant d'être utilisées.

  \item Avec chaque bécher, constituer une pile : l'une des demi-piles est constituée par le couple \ce{Cu^2+ / Cu}, l'autre par une électrode de référence que l'on fera tremper dans la solution.

  \item Pour chaque concentration, relever la différence de potentiels aux bornes de la pile constituée. Il faut commencer la série de mesures par la solution la moins concentrée (expliquer pourquoi).

    Le pont ionique est intégré à l'électrode de référence. De plus, par construction, le potentiel de cette électrode est fixe et vaut $E_\text{ref}$. On mesure $U=E-E_\text{ref}$. 

  \item Tracer le potentiel $E$ en fonction de $\log \left( \frac{c}{\cz} \right) $, où $c$ est la concentration de la solution de sulfate de cuivre. La théorie prévoit 
    \begin{equation}
      E = \num{0.34} + \num{0.03}\log\left( \frac{[\ce{Cu^2+}]}{\cz} \right) 
    \end{equation}
  Comparer cette expression aux valeurs expérimentales. Que vaut le potentiel standard du couple \ce{Cu^2+ / Cu} ?

  \item Déterminer le potentiel standard des couples \ce{Ag+ / Ag }, \ce{Pb^2+ / Pb} et \ce{Zn^2+ / Zn}.
\end{itemize}

Les électrodes de référence possibles sont :
\begin{itemize}
  \item électrode au calomel (fil rouge) $E_\text{ref}=\SI{0.241}{\volt}$ ;
  \item électrode d'argent (la plus probable) $E_\text{ref}=\SI{0.225}{\volt}$;
  \item électrode au sulfate mercureux  (la moins probable) $E_\text{ref}=\SI{0.651}{\volt}$. 
\end{itemize}

\end{document}

 
