%! TeX program = lualatex
\documentclass{tp}
\usepackage{pgfplots}
\pgfplotsset{compat=1.17}
\usetikzlibrary{decorations.text}
\titre{TP32 : Magnétisme}
\begin{document}
%\small


\section{Objectif du TP}
L'objectif de ce TP est de mettre en \oe{}uvre différentes méthodes de mesure de champs magnétiques, puis de produire un champ magnétique tournant afin de mettre en évidence son effet moteur sur un moment magnétique.

\section{Mesure du champ magnétique terrestre}%
\label{sec:mesure_du_champ_magnetique_terrestre}

Nous allons mesurer la composante horizontale du champ magnétique terrestre par la méthode des tangentes. 

On commence par placer une boussole horizontalement et on laisse sont aiguille s'orienter selon la composante horizontale $\vv*{B}{h}$ du champ magnétique terrestre local. On place ensuite une bobine plate centrée sur la bobine et dont l'axe est orienté perpendiculairement à $\vv*{B}{h}$.

On appelle $\alpha$ l'angle dont a tourné l'aiguille de la boussole lorsqu'on fait passer un courant $I$ dans la spire. On donne le champ magnétique créé au centre d'une bobine plate de rayon $R$ comportant $N$ spires et parcourue par un courant $I$ :
\begin{equation}
  \vv*{B}{b} = \frac{\mu_0 N I }{2R}\vv{n}
\end{equation}
où $\vv{n}$ est le vecteur unitaire normal au plan de la bobine, orienté par $I$. 

\begin{itemize}
  \item Mesurer l'angle $\alpha$ pour $I$ allant de \num{-5} à \SI{5}{\ampere}.
  \item Tracer la courbe $I = f(\tan(\alpha))$. En déduire une première valeur de $B_h$. 
  \item Mesurer l'angle $\alpha$ pour $N=1$ et $I=\SI{2}{\ampere}$ en fonction de $R$.  En déduire une seconde estimation de la valeur de $B_h$. 
  \item Sachant qu'en France, le champ magnétique terrestre a une inclinaison d'environ \SI{60}{\degree} avec l'horizontale, déterminer une estimation de la norme du champ magnétique terrestre total.
\end{itemize}

\section{Mesure du champ produit sur l'axe d'une bobine}%
\label{sec:mesure_du_champ_produit_sur_l_axe_d_une_bobine}

\subsection{Principe d'un teslamètre à effet Hall}%
\label{sub:principe_d_un_teslametre_a_effet_hall}

La sonde à effet Hall est une plaquette de semi-conducteur parcouru par un courant $I$. 

On s'intéresse au champ magnétique perpendiculaire à la plaquette. Les porteurs de charge sont soumis à une force de Lorentz $\vv{F} = q\vv{E} + q\vv{v}\wedge \vv{B}$, $\vv{v}$ étant la vitesse moyenne des porteurs de charge. Des charges s'accumulent sur les surfaces du conducteur (voir schéma) et à l'équilibre, elles ont un mouvement rectiligne uniforme, donc 
\begin{equation}
  \vv{F}=\vv{0} \Leftrightarrow \vv{E} = - \vv{v}\wedge \vv{B}
\end{equation}
Il existe donc une tension $U_H$ entre les deux faces chargées du semi-conducteur
\begin{equation}
  U_H = \int_1^2 \vv{v}\wedge \vv{B}\cdot \vv{\D \ell} = dvB
\end{equation} 
Si on note $h$ l'épaisseur du semi-conducteur et $n$ le nombre de porteurs de charge mobile par unité de volume, 
\begin{equation}
  I = nqvdh \quad \text{donc} \quad U_H = \frac{IB}{nqh}
\end{equation}
On remarque que $U_H$ est inversement proportionnel à $n$, c'est pour cette raison que l'effet Hall est plus visible dans un semi-conducteur.

\begin{center}
  \begin{tikzpicture}
    \def\d{3}
    \def\h{1}
    \def\l{4}
    \coordinate (A) at (0, \h/2, -\d/2);
    \coordinate (B) at (\l, \h/2, -\d/2);
    \coordinate (C) at (\l, \h/2, +\d/2);
    \coordinate (D) at (0, \h/2, +\d/2);
    \coordinate (E) at (0, -\h/2, -\d/2);
    \coordinate (F) at (\l, -\h/2, -\d/2);
    \coordinate (G) at (\l, -\h/2, +\d/2);
    \coordinate (H) at (0, -\h/2, +\d/2);
    \draw[dashed] (A) -- (E);
    \draw[] (B) -- (F);
    \draw[] (C) -- (G);
    \draw[] (D) -- (H);
    \draw[thick, postaction=decorate, decoration={markings, mark=at position 0.5 with {\arrow{latex}}}] (-2, 0,0) -- (0,0,0) node[midway, above]{$I$};
    \fill[white, fill opacity=0.8] (A) -- (E) -- (H) -- (D) -- cycle;
    \draw (A) -- (B)-- (C) -- (D) -- cycle;
    \draw[dashed](H) -- (E) --(F);
    \draw (F) -- (G) -- (H);
    \def\n{5}
    \fill[white, fill opacity=0.8] (C) -- (D) -- (H) -- (G) -- cycle;
    \draw (G) node[above left]{\small 1};
    \foreach \i in {1,2,...,\n}{
      \node[circle, draw, inner sep=2pt, fill=white] at ({\i*\l/(\n+1)}, 0, \d/2) {$+$};
      \node[circle, draw, inner sep=2pt, ] at ({\i*\l/(\n+1)}, 0, -\d/2) {$-$};
    }
    \draw[dashed] (A) -- ++(-2,0) coordinate (d1);
    \draw[dashed] (D) -- ++(-2,0) coordinate (d2);
    \draw[latex-latex] (d1) -- (d2) node[midway, fill=white,]{$d$}; 
    \draw[dashed] (B) -- ++(1,0) coordinate (h1);
    \draw[dashed] (F) -- ++(1,0) coordinate (h2);
    \draw[latex-latex] (h1) -- (h2) node[midway, fill=white,]{$h$}; 

    \fill[white, fill opacity=0.8] (B) -- (F) -- (G) -- (C) -- cycle;
    \draw[thick, postaction=decorate, decoration={markings, mark=at position 0.5 with {\arrow{latex}}}] (\l, 0,0) -- ++(2,0,0) node[midway, below]{$I$};
    \draw (E) node[above right]{\small 2};


    \draw[-latex, thick] (\l+4, -0.5, 0) -- ++(0, 1, 0) node[midway, right]{$\vv{B}$};
  \end{tikzpicture}
\end{center}
\subsection{Champ magnétique sur l'axe d'une bobine ou d'un solénoïde}%
\label{sub:champ_magnetique_sur_l_axe_d_une_bobine_ou_d_un_solenoide}

L'expression du champ d'un solénoïde d'axe $(Ox)$ ($O$ situé au milieu du solénoïde), de longueur $L$, comportant $n$ spires par unité de longueur parcouru par un courant $I$ s'écrit en un point $M$ de l'axe
\begin{equation}
  \vv{B}(M) = \frac{\mu_0 n I}{2}\left( \cos(\alpha_2) - \cos(\alpha_1) \right) \vex \quad \text{avec} \quad \cos(\alpha_2) = \frac{x+L/2}{\sqrt{(x+L/2)^2 +R^2}} \quad \text{et} \quad \cos(\alpha_1) = \frac{x-L/2}{\sqrt{(x-L/2)^2 +R^2}}
\end{equation}
\begin{center}
  \begin{tikzpicture}
  %tikz magnetisme
    \def\R{1.5}
    \def\L{6}
    \def\r{0.3}
    \draw (0, \R) coordinate(A) arc[start angle=90, end angle=270, x radius=\r, y radius=\R] coordinate (D);
    \draw[dashed] (0, -\R) arc[start angle=-90, end angle=90, x radius=\r, y radius=\R];
    \draw (A) -- ++(\L, 0) coordinate (B);
    \draw (\L, 0) ellipse[x radius=\r, y radius=\R];
    \draw (D) -- ++(\L, 0) coordinate (C);
    \draw[thick, postaction={decorate, decoration={markings, mark=at position 0.5 with{\arrow{latex};\node[left]{$I$};}}}] (\L/2, \R) arc[start angle=90, end angle=270, x radius=\r, y radius=\R] coordinate (D);
    \draw[dashed, thick] (\L/2, -\R) arc[start angle=-90, end angle=90, x radius=\r, y radius=\R];
    \draw[-latex] (0,0) -- (\L+4,0) node[right]{$x$ }; 
    \fill (\L+2, 0) circle(2pt) coordinate (M) node[above, yshift=2pt]{$M$ };
    \draw[dotted](M) -- (C);
    \draw[dotted](M) -- (A);
    \draw (\L+1, 0) arc (180:{180+atan(\R/2)}:1) node[midway, left]{$\alpha_1$ };
    \draw (\L-1, 0) arc (180:{180-atan(\R/(\L+2))}:3) node[midway, left]{$\alpha_2$ };
  \end{tikzpicture}
\end{center}
\begin{itemize}
  \item Faire les mesures pour \num{100} puis \num{200} spires. Tracer alors la courbe $B=f(x)$ et comparer à la théorie.
\end{itemize}
%\section{Champ magnétique tournant}%
%\label{sec:champ_magnetique_tournant}

%Nous allons maintenant créer un champ magnétique tournant à l'aide de deux bobines et mettre en rotation une aiguille aimantée. 

%\begin{itemize}
  %\item Réaliser le montage ci-dessous. Le GBF utilisé pour générer la tension sinusoïdale $u$ est le GF~467AF : utiliser la sortie "SORTIE AMPLI \SI{0.5}{\ohm}". Pour avoir un champ magnétique le plus intense possible, positionner les bobines très proches l'une de l'autre (quasiment accolées) perpendiculairement l'une à l'autre. On prendra pour $C$ la plus grande valeur possible avec les condensateurs fournis ($C=\SI{15.5}{\micro\farad}$). Le nombre de spires doit être le même pour les deux bobines.

  %\begin{center}
    %\begin{tikzpicture}
      %\draw (0,0) -- (0, 4) to[L=bobine 1] (2, 4) -- (2, 0);
      %\draw (0,1) to[short,*-] (2,1) to[C,l_=$C$ ] (4, 1) to[L, l_=bobine 2, mirror] (4, 3) to[short, -*] (2,3);
      %\draw[-latex] (0.2, 0) -- (1.8, 0) node[midway, below]{$u$};
    %\end{tikzpicture}
  %\end{center}

  %\item Avant d'allumer le GBF, vérifier que tous les potentiomètres sont au minimum. Alimenter alors les deux bobines par une tension sinusoïdale $u$. Régler la fréquence sur quelques \si{\hertz}.

  %\item Placer une aiguille aimantée (boussole) sur la cale en bois, au niveau du point $O$.  Qu'observez-vous ? Si elle n'arrive pas à accrocher le champ, augmenter légèrement la fréquence.

  %\item Faire varier significativement la fréquence $f$ d'alimentation du GBF. Qu'observez-vous ?

  %\item Revenir à une fréquence $f$ permettant de voir l'aiguille tourner de façon régulière.
%\end{itemize}
%Pour mesurer la fréquence de rotation de l'aiguille, nous allons utiliser un stroboscope. 
%\begin{itemize}
  %\item Brancher le stroboscope à la sortie d'un second GBF, et l'utiliser pour mesurer la fréquence de rotation de l'aiguille. La comparer à $f$. 

  %\item Supprimer la bobine 2. Les résultats obtenus sont-ils très différents ? 

  %\item Estimer l'impédance du condensateur à la fréquence utilisée pour faire ce montage. Le courant qui circule dans la bobine 2 est-il suffisant pour que le champ magnétique puisse réellement être considéré comme \textit{tournant} ?
%\end{itemize}
\end{document}

