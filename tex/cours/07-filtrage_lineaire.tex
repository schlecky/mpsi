\documentclass{cours}
\usepackage{pgfplots}
\usepackage{multicol}
\usepackage{pythontex}
\usepackage{calc}
\usetikzlibrary{patterns}
\begin{document}
\setcounter{chapter}{6}
\shorthandoff{:!}
\chapter{Filtrage linéaire}
\section{Signaux périodiques}
Un signal périodique $s(t)$ est composé d'un motif élémentaire de durée $T$ qui se répète indéfiniment au cours du temps.
\begin{center}
\begin{tikzpicture}
    %tikz elec
\begin{axis}[
  height=4cm,
  width=8cm,
  xmin=-1,xmax=2*2*pi,
  ymin=-1.5,ymax=1.5,
  xtick=\empty,
  ytick=\empty,
  axis y line=middle,
  axis x line=middle,
  clip=false,
  xlabel=$t$,
  ylabel=$s(t)$,
  ylabel style = {anchor=south,at={(0,1)} },
  xlabel style = {anchor=west},
  ]

\addplot[domain=0:2*2*pi,samples=200,smooth,coul1]{0.7*cos(3*deg(x))+0.5*cos(6*deg(x))};
\addplot[domain=2*pi:2.66*pi,samples=200,smooth,coul2]{0.7*cos(3*deg(x))+0.5*cos(6*deg(x))};
\draw[dotted] (axis cs:2*pi,1.5) -- (axis cs:2*pi,-1.5);
\draw[dotted] (axis cs:2.66*pi,1.5) -- (axis cs:2.66*pi,-1.5);
\draw[<->] (axis cs:2*pi,-1.5) -- (axis cs:2.66*pi,-1.5) node[midway,below]{$T$};
\end{axis}
\end{tikzpicture}
\end{center}
\begin{itemize}
\item $T$ est la \textbf{période} du signal ;
\item $f=\frac{1}{T}$ est sa fréquence.
\end{itemize}
On peut décomposer une fonction périodique en \textbf{série de Fourier} : 
\begin{equation*}
s(t) = a_0 + \sum_{n=1}^{\infty}a_n\cos(n\omega_0t) + b_n\sin(n\omega_0t)
\end{equation*}
avec 
\begin{equation*}
a_n=\frac{2}{T}\int_0^Ts(t)\cos(n\omega_0t)\,\D t \quad \text{et} \quad b_n = \frac{2}{T}\int_0^Ts(t)\sin(n\omega_0t)\,\D t.
\end{equation*}
$a_0$ est la valeur moyenne de $s(t)$ :
\begin{equation*}
a_0 = \frac{1}{T}\int_0^Ts(t)\D t
\end{equation*}

L'harmonique de rang $n$ est :
\begin{equation*}
H_n=a_n\cos(n\omega_0t) + b_n\sin(n\omega_0t) = c_n\cos(n\omega_0t+\varphi_n)
\end{equation*}

Un signal périodique se décompose en une somme d'une composante continue et d'harmoniques sinusoïdales de fréquences multiples de la fréquence du signal.

\begin{center}
\begin{tikzpicture}[baseline=0]
\draw[->] (-1,0) -- (5,0) node[right] {$t$};
\draw[->] (0,-1) -- (0,2) node[left]{$s(t)$};
\foreach \x in {0,...,4} {
  \draw[coul1] (\x,-0.5) --++ (0.7,1.5) --++ (0.3,-1.5);
}
\draw (2.5,-2.5) node[below]{Représentation temporelle};
\end{tikzpicture}
\hspace{1cm}
\begin{tikzpicture}[baseline=0]
\draw[->] (-1,0) -- (5,0) node[right] {$f$};
\draw[->] (0,-1) -- (0,1.5) node[left]{$A$};
\draw[coul1,thick] (0,0) node[below left,black]{$0$}-- (0,0.6);
\draw[coul1,thick] (1,0) node[below,black]{$f$}-- (1,1);
\draw[coul1,thick] (2,0) node[below,black]{$2f$}-- (2,0.7);
\draw[coul1,thick] (3,0) node[below,black]{$3f$}-- (3,0.4);
\draw[coul1,thick] (4,0) node[below,black]{$4f$}-- (4,0.2);
\draw (0,0) to[out=-90,in=180] (1,-1.5) node[right]{Composante continue};
\draw (1,-0.5) to[out=-90,in=180] (1,-1) node[right]{Fondamentale};
\draw[decorate, decoration=brace] (1,1.2) -- (4,1.2) node[midway,above] {Harmoniques};
\draw (2.5,-2.5) node[below]{Spectre};
\end{tikzpicture}
\end{center}


Le spectre du signal $s(t)$ est composé de l'ensemble des amplitudes des harmoniques (et de la composante continue).

La valeur moyenne d'un signal $s(t)$ de période $T$ est : 
\begin{equation*}
<s(t)> = \frac{1}{T}\int_0^Ts(t)\, \D t
\end{equation*}

La valeur efficace $S$ d'un signal $s(t)$ de période $T$ est :
\begin{equation*}
S = \sqrt{\frac{1}{T}\int_0^Ts^2(t)\,\D t}
\end{equation*}
par exemple pour un signal sinusoïdal $s(t) = A\sin(\omega t)$ on a :
\begin{equation}
<s(t)> = \frac{A}{T}\int_0^T\sin(\omega t)\, \D t=0
\end{equation}
et 
\begin{equation}
S=\sqrt{\frac{1}{T}\int_0^TA^2\sin^2(\omega t)\, \D t}=\frac{A}{\sqrt{T}}\sqrt{\int_0^T \frac{1}{2}(1-\cos 2t)\, \D t}=\frac{A}{\sqrt{2}}
\end{equation}

\section{Filtres}
Un filtre est un système linéaire qui transmet certaines fréquences et en atténue d'autres.

Un filtre est caractérisé par sa \textbf{fonction de transfert harmonique} $\underline{H}(\omega)=\frac{\underline{s}(\omega)}{\underline{e}(\omega)}$ dont le module est le \textbf{gain} du filtre à la pulsation $\omega$.


\paragraph{Filtres les plus courants :}
\begin{itemize}
\item Filtre passe-bas : laisse passer les basses fréquences ($\omega<\omega_c$) :
\begin{center}
\begin{tikzpicture}[baseline=0pt]
\draw[->] (0,0) -- (0,1.5) node[above]{$|H(\omega)|$};
\draw[->] (0,0) -- (3,0) node[right] {$\omega$};
\draw[coul1] (0,1) -- (1.5,1) -- (1.5,0) -- (3,0);
\draw[coul2] (0,1) edge[out=0,in=180] (3,0); 
\draw (1.5,0) node[below]{$\omega_c$};
\draw (0,1) node[left]{1};
\draw (0.75,1) node[above,align=center]{idéal};
\draw (2,0.5) node[right] {réel};
\end{tikzpicture}
\end{center}

\item Filtre passe-haut : laisse passer les hautes fréquences ($\omega>\omega_c$) :
\begin{center}
\begin{tikzpicture}[baseline=0pt]
\draw[->] (0,0) -- (0,1.5) node[above]{$|H(\omega)|$};
\draw[->] (0,0) -- (3,0) node[right] {$\omega$};
\draw[coul1] (0,0) -- (1.5,0) -- (1.5,1) -- (3,1);
\draw[coul2] (0,0) edge[out=0,in=180] (3,1); 
\draw (1.5,0) node[below]{$\omega_c$};
\draw (0,1) node[left]{1};
\draw (2.25,1) node[above,align=center]{idéal};
\draw (1,0.5) node[left] {réel};
\end{tikzpicture}
\end{center}

\item Filtre passe-bande : laisse passer les fréquences intermédiaires ($\omega_{c1}<\omega<\omega_{c2}$) :
\begin{center}
%\hspace{\stretch{1}}
\begin{tikzpicture}[baseline=0pt]
\draw[->] (0,0) -- (0,1.5) node[above]{$|H(\omega)|$};
\draw[->] (0,0) -- (3,0) node[right] {$\omega$};
\draw[coul1] (0,0) -- (1,0) -- (1,1) -- (2,1) -- (2,0) -- (3,0);
\draw[coul2] (0,0) to[out=0,in=180] (1.5,1) to[out=0,in=180] (3,0); 
\draw (1,0) node[below]{$\omega_{c1}$};
\draw (2,0) node[below]{$\omega_{c2}$};
\draw (0,1) node[left]{1};
\draw (2,1) node[above right]{idéal};
\draw (2.5,0.5) node[right] {réel};
\end{tikzpicture}
\end{center}
\end{itemize}

On définit le \textbf{gain} à la pulsation $\omega$ par $G(\omega)=|H(\omega)|$

On définit le \textbf{gain en décibel} à la pulsation $\omega$ par $G_{dB}(\omega)=20\log(|H(\omega)|)$.

\section{Exemple d'un filtre passe-bas d'ordre 1}
On étudie le filtre suivant :
\begin{center}
\begin{tikzpicture}
\draw (0,0) to[european resistor,l=$R$,i>^=$\underline{i}$] (3,0) to[C=$C$] (3,-2) to[short] (0,-2);
\draw (3,0) to[short,i=\mbox{$i=0$}] (4,0);
\draw (3,-2) -- (4,-2);
\draw[->] (0,-1.9) -- (0,-0.1) node[midway,left]{$\underline{e}(\omega)$};
\draw[->] (4,-1.9) -- (4,-0.1) node[midway,right]{$\underline{s}(\omega)$};
\end{tikzpicture}
\end{center}

\paragraph{Analyse qualitative :}
\begin{itemize}
\item Lorsque $\omega\rightarrow\infty$ le condensateur se comporte comme un fil et $s(\omega)\rightarrow 0$.
\item Lorsque $\omega\rightarrow 0 $ le condensateur se comporte comme un interrupteur ouvert et $s(\omega)\rightarrow e(\omega)$.
\end{itemize}
Ce filtre est donc un \textbf{filtre passe-bas}.

\paragraph{Analyse quantitative :}
La tension de sortie est $\underline{s}(\omega)=\dfrac{1/jC\omega}{R+1/jC\omega}\underline{e}(\omega)$ (pont diviseur de tension). Soit $\underline{s}(\omega)=\dfrac{1}{1+jRC\omega}\underline{e}(\omega)$. Donc la fonction de transfert du filtre est :
\begin{equation*}
\underline{H}(\omega)=\frac{1}{1+jRC\omega} = \frac{1}{1+j \frac{\omega}{\omega_0}}.
\end{equation*}

Le \textbf{gain} du filtre est :
\begin{equation*}
G=|\underline{H}(\omega)| = \frac{1}{\sqrt{1+(\omega/\omega_0)^2}}
\end{equation*}

Le \textbf{gain en décibel} est :
\begin{equation*}
G_{dB}=20\log(G) = -20\log\sqrt{1+(\omega/\omega_0)^2}
\end{equation*}

On trace $G_{dB}$ en fonction de $\log\left(\frac{\omega}{\omega_0}\right)$, c'est le \textbf{diagramme de Bode} du filtre.

\begin{center}
\begin{tikzpicture}
    %tikz elec
\begin{axis}[
  height=4cm,
  width=8cm,
  xmin=0.1,xmax=100,
  ymin=-40,ymax=5,
  xmode=log,
  xtick=\empty,
  ytick=\empty,
  axis y line=middle,
  axis x line=middle,
  clip=false,
  xlabel=$\log\left(\frac{\omega}{\omega_0}\right)$,
  ylabel=$G_{dB}$,
  ylabel style = {anchor=south,},
  xlabel style = {anchor=west},
  ]
\addplot[domain=0.05:100,samples=200,smooth,thick,coul1]{-20*ln(sqrt(1+x^2))/ln(10)};
\addplot[domain=0.05:1, dashed, thick]{0};
\addplot[domain=1:100, dashed, thick]{-20*log10(x)} node[pos=0.7, above, sloped]{\SI{-20}{dB/décade}};
\draw (axis cs:1,0) node[above right] {$0$};
\draw ($(axis cs:1,-3)+(-0.1, 0)$) node[below left]{\SI{-3}{dB}}--++ (0.2,0) ;
\end{axis}
\end{tikzpicture}
\end{center}

Le diagramme de Bode fait apparaître des zones rectilignes. Cherchons les asymptotes en $\pm\infty$ (pour $\log(\omega)$) :
\begin{itemize}
\item $\omega\gg\omega_0$ : $G_{dB}=-20\log\sqrt{1+(\omega/\omega_0)^2}\simeq -20\log \frac{\omega}{\omega_0} \Leftrightarrow y=ax$

L'asymptote en $+\infty$ est donc une doite de pente -20dB/décade, c'est à dire que lorsque l'on multiplie $\omega$ par 10, $G_{dB}$ diminue de 20\,dB.

\item $\omega\ll\omega_0$ : $G_{dB}\rightarrow 0$ on a donc une asymptote horizontale lorsque $\omega\rightarrow 0$ (ou $\log\omega\rightarrow -\infty$).
\end{itemize}

La phase de la fonction de transfert est :
\[\varphi = -\arctan\left(\frac{\omega}{\omega_0}\right) \]

\begin{center}
\begin{tikzpicture}
    %tikz elec
\begin{axis}[
  height=4cm,
  width=8cm,
  xmin=0.05,xmax=100,
  ymin=-pi/2,ymax=0.2,
  xmode=log,
  xtick=\empty,
  ytick=\empty,
  axis y line=middle,
  axis x line=middle,
  clip=false,
  xlabel=$\log\left(\frac{\omega}{\omega_0}\right)$,
  ylabel=$\varphi$,
  ylabel style = {anchor=south,},
  xlabel style = {anchor=west},
  ]
\addplot[domain=0.05:100,samples=200,smooth,thick,coul1]{rad(-atan(x))};
\draw[thick, dashed] (axis cs:1, -pi/2) node[left]{$-\frac{\pi}{2}$} -- (axis cs:100, -pi/2);
\draw[thick, dashed] (axis cs:1, -pi/4) node[right]{$-\frac{\pi}{4}$} ;
\draw[thick, dashed] (axis cs:0.05, 0) -- (axis cs:1, 0);
%\addplot[domain=0.05:1, dashed, thick]{0};
%\addplot[domain=1:100, dashed, thick]{-20*log10(x)} node[pos=0.7, above, sloped]{\SI{-20}{dB/décade}};
\end{axis}
\end{tikzpicture}
\end{center}

\paragraph{Comportement intégrateur :}%
\label{par:comportement_integrateur_}
Intéressons-nous à ce qu'il se passe pour $\omega\gg\omega_0$. La fonction de transfert du filtre s'écrit :
\[ \ul{H}(\omega) \approx \frac{1}{jRC\omega} \]

et donc le signale de sortie est :
\[ \ul{s}(\omega) = \frac{1}{RC} \frac{\ul{e}(\omega)}{j\omega}\]

Or, $\frac{\ul{e}}{j\omega}$  correspond à la primitive de $\ul{e}(\omega)$. Dans ce domaine de fréquences, le filtre a un comportement d'intégrateur.

\paragraph{Utilisation comme moyenneur}%
\label{par:utilisation_comme_moyenneur}
Si la fréquence de coupure $f_0 = \omega_0/2\pi$  du filtre passe-bas est très inférieure à la fréquence fondamentale d'un signal périodique non sinusoïdal, seule la composante continue (valeur moyenne) du signal sera transmise par le filtre. À la sortie du filtre, on retrouve la valeur moyenne du signal d'entrée, on a réalisé un \emph{moyenneur}.
\begin{center}
\begin{tikzpicture}[baseline=0]
\draw[->] (-1,0) -- (5,0) node[right] {$f$};
\draw[->] (0,-1) -- (0,1.5) node[left]{$A$};
\draw[coul1,thick] (0,0) node[below left,black]{$0$}-- (0,0.6);
\draw[coul1,thick] (1,0) node[below,black]{$f$}-- (1,1);
\draw[coul1,thick] (2,0) node[below,black]{$2f$}-- (2,0.7);
\draw[coul1,thick] (3,0) node[below,black]{$3f$}-- (3,0.4);
\draw[coul1,thick] (4,0) node[below,black]{$4f$}-- (4,0.2);
\draw[thick, dashed](0.2,0) node[below] {$f_0$ }-- (0.2,1.5) ;
\draw (0,0) to[out=-90,in=180] (1,-1.5) node[right]{Valeur moyenne};
\draw (1,-0.5) to[out=-90,in=180] (1,-1) node[right]{Fondamentale};
\end{tikzpicture}
\end{center}


\section{Autres filtres passifs}
\subsection{Filtre passe-haut d'ordre 1}
\begin{center}
\begin{tikzpicture}
\draw (0,0) to[C=$C$,i>^=$\underline{i}$] (3,0) to[european resistor,l=$R$] (3,-2) to[short] (0,-2);
\draw (3,0) to[short,i=\mbox{$i=0$}] (4,0);
\draw (3,-2) -- (4,-2);
\draw[->] (0,-1.9) -- (0,-0.1) node[midway,left]{$\underline{e}(\omega)$};
\draw[->] (4,-1.9) -- (4,-0.1) node[midway,right]{$\underline{s}(\omega)$};
\end{tikzpicture}
\end{center}
\begin{itemize}
\item Lorsque $\omega \rightarrow 0$ le condensateur se comporte comme un interrupteur ouvert et $s(\omega)\rightarrow 0$
\item Lorsque $\omega \rightarrow \infty$ le condensateur se comporte comme un fil et $s(\omega) \rightarrow e(\omega)$.
\end{itemize}
On construit donc bien un filtre passe-haut de cette manière.

La fonction de transfert de ce filtre est (pont diviseur de tension):
\begin{equation*}
H(\omega)=\frac{s(\omega)}{e(\omega)} = \frac{R}{R+1/jC\omega} = \frac{1}{1+(jRC\omega)^{-1}} = \frac{1}{1-j \frac{\omega_0}{\omega}} \quad \text{avec} \quad \omega_0=\frac{1}{RC}
\end{equation*}
Le gain du filtre est $G=|H(\omega)| = \dfrac{1}{\sqrt{1+\left( \frac{\omega_0}{\omega} \right)^2}}$ et $G_{dB}=-20\log\sqrt{1+\left( \frac{\omega_0}{\omega} \right)^2}$

\begin{center}
\begin{tikzpicture}
\begin{axis}[
  height=6cm,
  width=8cm,
  xmin=0.1,xmax=10,
  ymin=-20,ymax=5,
  xmode=log,
  xtick=\empty,
  ytick=\empty,
  axis y line=middle,
  axis x line=middle,
  clip=false,
  xlabel=$\log(\frac{\omega}{\omega_0})$,
  ylabel=$G_{dB}$,
  ylabel style = {anchor=south,},
  xlabel style = {anchor=west},
  ]
    %tikz elec
\addplot[domain=0.1:10,samples=200,smooth,thick,coul1]{-20*ln(sqrt(1+(1/x)^2))/ln(10)};
\addplot[domain=0.1:1,samples=200,smooth,thick, dashed]{20*log10(x)};
\addplot[domain=1:10,samples=200,smooth,thick, dashed]{0};
\draw (axis cs:1,0) node[above right] {$0$};
\draw (axis cs:1,-3) --++ (0.2,0) node[right]{\SI{-3}{dB}};
\draw (axis cs:0.2,-12) node[rotate=47] {20 dB/décade};
\end{axis}
\end{tikzpicture}
\end{center}

La phase de la fonction de transfert est 
\[
\varphi = \arctan\left( \frac{\omega_0}{\omega} \right) 
\]

\begin{center}
\begin{tikzpicture}
\begin{axis}[
  height=4cm,
  width=8cm,
  xmin=0.01,xmax=100,
  ymin=0,ymax=pi/2+0.3,
  xmode=log,
  xtick=\empty,
  ytick=\empty,
  axis y line=middle,
  axis x line=middle,
  clip=false,
  xlabel=$\log\left(\frac{\omega}{\omega_0}\right)$,
  ylabel=$\varphi$,
  ylabel style = {anchor=south,},
  xlabel style = {anchor=west},
  ]
\addplot[domain=0.01:100,samples=200,smooth,thick,coul1]{rad(atan(1/x))};
\draw[thick, dashed] (axis cs:1, pi/2) node[right]{$\frac{\pi}{2}$} -- (axis cs:0.01, pi/2);
\draw[thick, dashed] (axis cs:1, pi/4) node[right]{$\frac{\pi}{4}$} ;
\draw[thick, dashed] (axis cs:1, 0) -- (axis cs:100, 0);
%\addplot[domain=0.05:1, dashed, thick]{0};
%\addplot[domain=1:100, dashed, thick]{-20*log10(x)} node[pos=0.7, above, sloped]{\SI{-20}{dB/décade}};
\end{axis}
\end{tikzpicture}
\end{center}

\paragraph{Comportement dérivateur : }%
\label{par:comportement_derivateur_}
Pour des pulsations $\omega\gg\omega_0$, la fonction de transfert peut s'écrire 
\[
\ul{H}(\omega) \approx jRC\omega
\]
et la tension de sortie devient :
\[
\ul{s}(\omega) = RC\times j\omega\ul{e}(\omega)
\]
$j\omega\ul{e}(\omega)$ correspond à la dérivée temporelle du signal d'entrée. Dans ce domaine de fréquences, le filtre a un comportement dérivateur. 

\subsection{Filtre passe-bas d'ordre 2}
\begin{center}
\begin{tikzpicture}
\draw (0,0) to[european resistor,l=$R$] (3,0) to[L=$L$] (6,0) to[C,l=$C$] (6,-2) to[short] (0,-2);
\draw (6,0) to[short,i=\mbox{$i=0$}] (7,0);
\draw (6,-2) -- (7,-2);
\draw[->] (0,-1.9) -- (0,-0.1) node[midway,left]{$\underline{e}(\omega)$};
\draw[->] (7,-1.9) -- (7,-0.1) node[midway,right]{$\underline{s}(\omega)$};
\end{tikzpicture}
\end{center}

\begin{itemize}
\item Lorsque $\omega \rightarrow \infty$ la bobine se comporte comme un interrupteur ouvert et le condensateur comme un fil. Donc $s(\omega)\rightarrow 0$ ; 
\item Lorsque $\omega \rightarrow 0$ la bobine se comporte comme un fil est le condensateur comme un interrupteur ouvert. Donc Donc $s(\omega)\rightarrow e(\omega)$.
\end{itemize}
La fonction de transfert du filtre est (pont diviseur de tension) : 
\begin{equation*}
H(\omega) = \frac{Z_C}{Z_C+Z_L+Z_R} = \frac{1/jC\omega}{R+j(L\omega-1/C\omega)} = \frac{1}{jRC\omega-LC\omega^2+1} = \frac{1}{1-\frac{\omega^2}{\omega_0^2}+j \frac{\omega}{\omega_0Q}}.
\end{equation*}
Le gain du filtre est $G = \dfrac{1}{\sqrt{\left( 1- \left( \frac{\omega}{\omega_0} \right)^2 \right)^2+\left( \frac{\omega}{Q\omega_0} \right)^2}}$. Et $G_{dB}=-10\log\left(\left( 1- \left( \dfrac{\omega}{\omega_0} \right)^2 \right)^2+\left( \dfrac{\omega}{Q\omega_0} \right)^2\right)$

\begin{center}
\begin{tikzpicture}
    %tikz elec
\begin{axis}[
  height=6cm,
  width=8cm,
  xmin=0.1,xmax=10,
  ymin=-40,ymax=20,
  xmode=log,
  xtick=\empty,
  ytick=\empty,
  axis y line=middle,
  axis x line=middle,
  clip=false,
  xlabel=$\log\left(\frac{\omega}{\omega_0}\right)$,
  ylabel=$G_{dB}$,
  ylabel style = {anchor=south,},
  xlabel style = {anchor=west},
  ]
\addplot[domain=0.1:10,samples=200,smooth,coul1, thick]{-10*ln((1-(x)^2)^2+x^2/0.25)/ln(10)};
\addplot[domain=0.1:10,samples=200,smooth,coul2, thick]{-10*ln((1-(x)^2)^2+x^2/25)/ln(10)};
\addplot[domain=1:10,samples=200,smooth, dashed, thick]{-40*log10(x)};
%\draw (axis cs:1.5,5) node[above right] {$\log(\omega_0)$};
\draw (axis cs:0.6,7) node[left] {$Q=5$};
\draw (axis cs:0.6,-7) node[left] {$Q=0.5$};
\draw (axis cs:8,-30) node[right]{pente = \SI{-40}{dB/décade}};
\end{axis}
\end{tikzpicture}
\end{center}

Pour des facteurs de qualité élevés ($Q>\frac{1}{\sqrt{2}}$ ), on observe un phénomène de résonance, autour de $\omega=\omega_0$, le gain devient supérieur à $1$  et le gain en décibels devient positif.

\subsection{Filtre passe-bande d'ordre 2}
\begin{center}
\begin{tikzpicture}
\draw (0,0) to[L,l=$L$] (3,0) to[C=$C$] (6,0) to[european resistor,l=$R$] (6,-2) to[short] (0,-2);
\draw (6,0) to[short,i=\mbox{$i=0$}] (7,0);
\draw (6,-2) -- (7,-2);
\draw[->] (0,-1.9) -- (0,-0.1) node[midway,left]{$\underline{e}(\omega)$};
\draw[->] (7,-1.9) -- (7,-0.1) node[midway,right]{$\underline{s}(\omega)$};
\end{tikzpicture}
\end{center}
\begin{itemize}
\item Lorsque $\omega \rightarrow \infty$ la bobine se comporte comme un interrupteur ouvert et le condensateur comme un fil. Donc $s(\omega)\rightarrow 0$ ; 
\item Lorsque $\omega \rightarrow 0$ la bobine se comporte comme un fil est le condensateur comme un interrupteur ouvert. Donc Donc $s(\omega)\rightarrow 0$.
\end{itemize}
On construit donc ainsi un filtre passe-bande. 

On trouve la fonction de transfert du filtre : $H(\omega)=\dfrac{1}{1+jQ \left( \frac{\omega}{\omega_0}-\frac{\omega_0}{\omega} \right)}$ avec $\omega_0=\dfrac{1}{\sqrt{LC}}$ et $Q=\dfrac{1}{R} \sqrt{\dfrac{L}{C}}$. 

Le gain est $G(\omega)=\dfrac{1}{\sqrt{1+Q^2\left( \frac{\omega}{\omega_0}-\frac{\omega_0}{\omega} \right)^2}}$, et $G_{dB}=-10\log \left( 1+Q^2\left( \frac{\omega}{\omega_0}-\frac{\omega_0}{\omega} \right)^2 \right)$

\begin{center}
\begin{tikzpicture}
    %tikz elec
\begin{axis}[
  height=6cm,
  width=8cm,
  xmin=0.1,xmax=10,
  ymin=-30,ymax=5,
  xmode=log,
  xtick=\empty,
  ytick=\empty,
  axis y line=middle,
  axis x line=middle,
  clip=false,
  xlabel=$\log\left(\frac{\omega}{\omega_0}\right)$,
  ylabel=$G_{dB}$,
  ylabel style = {anchor=south,},
  xlabel style = {anchor=west},
  ]
\addplot[domain=0.1:10,samples=200,smooth,coul3, thick]{-10*ln(1+0.1*(x-1/x)^2)/ln(10)};
\addplot[domain=0.1:10,samples=200,smooth,coul1, thick]{-10*ln(1+(x-1/x)^2)/ln(10)};
\addplot[domain=0.1:10,samples=200,smooth,coul2, thick]{-10*ln(1+9*(x-1/x)^2)/ln(10)};
%\draw (axis cs:1,0) node[above right] {$\log(\omega_0)$};
\draw (axis cs:10,-10) node[right]{$Q=0.3$};
\draw (axis cs:10,-20) node[right]{$Q=1$};
\draw (axis cs:10,-30) node[right]{$Q=3$};
%\draw (axis cs:0.6,7) node[left] {$Q=5$};
%\draw (axis cs:0.6,-7) node[left] {$Q=0.5$};
\end{axis}
\end{tikzpicture}
\end{center}

La courbe représentant $G_{dB}(\log(\omega))$ présente une asymptote en $-\infty$ de pente égale à 20 dB/décade et en $+\infty$ une asymptote de pente égale à 20 dB/décade.

On peut trouver la \emph{bande passante à -3~dB} en trouvant les valeurs $\omega_{c1}$ et $\omega_{c2}$ de $\omega$ pour lesquelles $G_{dB}(\omega)=-3 \Leftrightarrow G(\omega)=\frac{1}{\sqrt{2}}$. On trouve finalement
\begin{equation*}
\Delta\omega = \omega_{c2}-\omega_{c1} = \frac{\omega_0}{Q}
\end{equation*} 

\section{Mise en cascade de filtres}%
\label{sec:mise_en_cascade_de_filtres}

\subsection{Impédance d'entrée et de sortie}%
\label{sub:impedance_d_entree_et_de_sortie}

On peut modéliser un filtre de la manière suivante :

\begin{center}
  \begin{tikzpicture}
    \draw (0,0) to[short]  (2, 0) to[R=$Z_e$] (2, 3) to[short, i<=$\ul{i}_e$] (0, 3); 
    \draw (6,0) -- (4, 0) to[V=$\ul{s}$] (4, 1.5) to[R=$Z_s$] (4, 3) to[short, i=$\ul{i}_s$] (6,3); 
    \draw[-Stealth] (0, 0.2) -- (0,2.8) node[midway, left]{$\ul{e}$}; 
    \draw[-Stealth] (6, 0.2) -- (6,2.8) node[midway, right]{$\ul{s'}$}; 
    \draw[dashed](1.3, -0.5) rectangle (4.7, 3.5);
    \draw (3, -0.5) node[below]{filtre};
  \end{tikzpicture}
\end{center}

L'\textbf{impédance d'entrée} du filtre est $Z_e$ et l'\textbf{impédance de sortie} du filtre est $Z_s$. La fonction de transfert harmonique est toujours calculée avec $\ul{i}_s=0$.  On a donc 
\begin{equation}
  \ul{H} = \frac{\ul{s'}}{\ul{e}} = \frac{\ul{s}}{\ul{e}}
\end{equation}

\subsection{Mise en cascade de filtres}%
\label{sub:mise_en_cascade_de_filtres}

Il arrive qu'on mette deux filtres en cascade :

\begin{center}
  \begin{tikzpicture}
    \draw (0,0) to[short]  (2.2, 0) to[R=$Z_{e1}$] ++(0, 3) to[short, i<=$\ul{i}_{e1}$] ++(-2.2, 0); 
    \draw (6,0) -- (4, 0) to[V=$\ul{s}_1$] (4, 1.5) to[R=$Z_{s1}$] (4, 3) to[short, i=$\ul{i}_{s1}$] (6,3); 
    \draw[-Stealth] (0.5, 0.2) -- ++(0,2.6) node[midway, left]{$\ul{e}_1$}; 
    \draw[-Stealth] (5.5, 0.2) -- ++(0,2.6) node[midway, left]{$\ul{s'}_1$}; 
    \draw[dashed](1.3, -0.5) rectangle (4.7, 3.5);
    \draw (3, -0.5) node[below]{filtre 1};

    \begin{scope}[xshift=6cm]
      \draw (0,0) to[short]  (2.2, 0) to[R=$Z_{e2}$] ++(0, 3) to[short, i<=$\ul{i}_{e2}$] ++(-2.2, 0); 
      \draw (6,0) -- (4, 0) to[V=$\ul{s}_2$] (4, 1.5) to[R=$Z_{s2}$] (4, 3) to[short, i=$\ul{i}_s$] (6,3); 
    \draw[-Stealth] (0.5, 0.2) -- ++(0,2.6) node[midway, right]{$\ul{e}_2$}; 
    \draw[-Stealth] (5.5, 0.2) -- ++(0,2.6) node[midway, right]{$\ul{s'}_2$}; 
    \draw[dashed](1.3, -0.5) rectangle (4.7, 3.5);
    \draw (3, -0.5) node[below]{filtre 2};
    \end{scope}
  \end{tikzpicture}
\end{center}
Dans ces conditions, on n'a pas (alors que qu'on aimerait bien avoir !) $\ul{H}_\text{tot} = \ul{H}_1 \times \ul{H}_2$. La raison étant que comme $\ul{i}_{s1} \neq 0$, $\ul{s'}_1 \neq \ul{s}_1$.

On peut régler ce problème en concevant des filtres tels que $Z_{s1} \ll Z_{e2}$. 

\begin{loi}{Mise en cascade de filtres}
  Pour pouvroir mettre des filtres en cascade, on préférera des filtres dont l'impédance de sortie est faible et l'impédance d'entrée est élevée
  \begin{equation}
    Z_\text{entrée du filtre 2} \gg Z_\text{sortie du filtre 1}
  \end{equation}

  Dans ces conditions, la fonction de transfert totale du filtre est le produit des fonctions de transfert des filtres individuels.

  \begin{equation}
    \ul{H}_\text{tot} = \ul{H}_1 \times \ul{H}_2
  \end{equation}
\end{loi}

\section{Action d'un filtre sur un signal périodique}%
\label{sec:action_d_un_filtre_sur_un_signal_periodique}

Nous allons voir comment simuler l'action d'un filtre sur un signal périodique dont le spectre est connu. On s'intéresse à un signal carré $e(t)$ que l'on peut décomposer en série de Fourrier de la manière suivante

\begin{equation}
  e(t) = A\sum_{n=0}^\infty a_n\cos(n\omega_0t + \varphi_n) \quad \text{avec} \quad a_n = 
  \begin{cases}
    \frac{1}{n} \quad \text{si $n$ est impair} \\
    0 \quad \text{si $n$ est pair}
  \end{cases} \quad \text{et $\varphi_n = \frac{\pi}{2}$ }
\end{equation}

Pour déterminer l'action d'un filtre sur ce signal, on regarde comme le filtre modifie chaque composante de la décomposition en série de Fourrier, puis on reconstitue le signal. 

On commence par créer des listes contenant les $a_n$ et les $\varphi_n$ :
\begin{minted}{python}
import numpy as np
import matplotlib.pyplot as plt

N=1000    # On fera les calculs sur les 1000 premières harmoniques
w0=1      # On choisit arbitrairement ω0 = 1

# Création des listes d'amplitudes et de phases pour le signal carré
# les deux listes ont N+1 valeurs car il y a en plus la composante continue
ampl_carre = np.zeros(N)
phi_carre = np.zeros(N) + np.pi/2
for i in range(1, N):
    if i%2==1:
        ampl_carre[i] = 1/i
\end{minted}

On écrit ensuite une fonction \texttt{signal(ampl, phi, t)} qui reconstitue le signal aux instants \texttt{t}.

\begin{minted}{python}


# Reproduit le signal périodique défini par les harmoniques dont 
# les amplitudes sont dans la liste 'ampl' et les phases sont dans 
# la liste phi. w est la pulstation fondamentale du signal, t est la liste
# des temps auxquels le signal est calculé.
def signal(ampl, phi, w ,t):
    sortie = np.zeros(len(t))
    sortie += ampl[0]
    for i in range(1, len(ampl)):
        sortie += ampl[i]*np.cos(i*w*t + phi[i])
    return sortie
\end{minted}

Ensuite on écrit deux fonction qui décrivent l'action d'un filtre passe-bas d'ordre 1 et d'un filtre passe-haut d'ordre 1 sur les différentes harmoniques:
\begin{minted}{python}
# Passe-bas d'ordre 1
# wc est la pulsation de coupure du filtre
def filtre_pb1(ampl, phi, w, wc):
    # fonction de transfert H(w) = 1/(1+jw/wc)
    ampl_s = np.zeros(len(ampl))
    phi_s = np.zeros(len(phi))
    for i in range(N):
        ampl_s[i] = ampl[i]/np.sqrt(1+(i*w/wc)**2)
        phi_s[i]  = phi[i] - np.arctan(i*w/wc)
    return ampl_s, phi_s

# Passe-haut d'ordre 1
# wc est la pulsation de coupure du filtre
def filtre_ph1(ampl, phi, w, wc):
    # fonction de transfert H(w) = 1/(1-jwc/w)
    ampl_s = np.zeros(len(ampl))
    phi_s = np.zeros(len(phi))
    ampl_s[0] = 0
    for i in range(1, N):
        ampl_s[i] = ampl[i]/np.sqrt(1+(wc/(i*w))**2)
        phi_s[i]  = phi[i] + np.arctan(i*wc/(i*w))
    return ampl_s, phi_s
\end{minted}

On peut finalement observer l'effet du filtre sur le signal carré. On commence par appliquer le filtre passe-haut avec une pulsation de coupure égale à $\omega_0$. 

\begin{minted}{python}
t = np.linspace(0, 20, 200)
# Le signal carré d'origine
s = signal(ampl_carre, phi_carre, t)

# On calcule l'effet du filtre
ampl_f, phi_f = filtre_ph1(ampl_carre, phi_carre, w, w0)
s_f = signal(ampl_f, phi_f, t)

# On affiche le signal d'origine et le signal filtré
plt.plot(t,s)
plt.plot(t,s2)
plt.show()
\end{minted}

\begin{pycode}
import numpy as np
import matplotlib.pyplot as plt
import tikzplotlib

N=1000    # On fera les calculs sur les 1000 premières harmoniques
w0=1      # On choisit arbitrairement ω0 = 1


ampl_carre = np.zeros(N)
phi_carre = np.zeros(N) + np.pi/2
for i in range(1, N):
    if i%2==1:
        ampl_carre[i] = 1/i

# Reproduit le signal périodique défini par les harmoniques dont 
# les amplitudes sont dans la liste 'ampl' et les phases sont dans 
# la liste phi. w est la pulstation fondamentale du signal, t est la liste
# des temps auxquels le signal est calculé.
def signal(ampl, phi, w ,t):
    sortie = np.zeros(len(t))
    sortie += ampl[0]
    for i in range(1, len(ampl)):
        sortie += ampl[i]*np.cos(i*w*t + phi[i])
    return sortie

# Passe-bas d'ordre 1
# wc est la pulsation de coupure du filtre
def filtre_pb1(ampl, phi, w, wc):
    # fonction de transfert H(w) = 1/(1+jw/wc)
    ampl_s = np.zeros(len(ampl))
    phi_s = np.zeros(len(phi))
    for i in range(N):
        ampl_s[i] = ampl[i]/np.sqrt(1+(i*w/wc)**2)
        phi_s[i]  = phi[i] - np.arctan(i*w/wc)
    return ampl_s, phi_s

# Passe-haut d'ordre 1
# wc est la pulsation de coupure du filtre
def filtre_ph1(ampl, phi, w, wc):
    # fonction de transfert H(w) = 1/(1-jwc/w)
    ampl_s = np.zeros(len(ampl))
    phi_s = np.zeros(len(phi))
    ampl_s[0] = 0
    for i in range(1, N):
        ampl_s[i] = ampl[i]/np.sqrt(1+(wc/(i*w))**2)
        phi_s[i]  = phi[i] + np.arctan(i*wc/(i*w))
    return ampl_s, phi_s

t = np.linspace(0, 20, 200)
# Le signal carré d'origine
s = signal(ampl_carre, phi_carre, w0, t)

# On calcule l'effet du filtre
ampl_f, phi_f = filtre_ph1(ampl_carre, phi_carre, w0, w0)
s_f = signal(ampl_f, phi_f, w0, t)

# On affiche le signal d'origine et le signal filtré
plt.plot(t,s)
plt.plot(t,s_f)
\end{pycode}

\begin{center}
  \begin{pycode}
print(tikzplotlib.get_tikz_code(axis_height="7cm", axis_width="10cm"))
plt.clf()
plt.cla()
  \end{pycode}
\end{center}

On peut aussi, par exemple, observer l'effet intégrateur du filtre passe bas, lorsque la pulsation de coupure est petite devant la pulsation du signal :

\begin{minted}{python}
# On calcule l'effet du filtre
ampl_f, phi_f = filtre_pb1(ampl_carre, phi_carre, w0 ,w0/10)
s_f = signal(ampl_f, phi_f, w0, t)

# On affiche le signal d'origine et le signal filtré
plt.plot(t,s)
plt.plot(t,5*s_f) # on amplifie le signal de sortie pour que ça soit plus clair
plt.show()
\end{minted}

\begin{center}
  \begin{pycode}
# On calcule l'effet du filtre
ampl_f, phi_f = filtre_pb1(ampl_carre, phi_carre, w0, w0/10)
s_f = signal(ampl_f, phi_f, w0, t)

# On affiche le signal d'origine et le signal filtré
plt.plot(t,s)
plt.plot(t,5*s_f)
print(tikzplotlib.get_tikz_code(axis_height="7cm", axis_width="10cm"))
  \end{pycode}
\end{center}
\end{document}
