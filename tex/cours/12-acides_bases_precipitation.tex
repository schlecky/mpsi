\documentclass{cours}
\usepackage{pgfplots}
\usepackage{multicol}
\usepackage{cases}
\usepackage{amssymb}
\usepackage{calc}
%\usepackage{esvect}
\usepackage{mhchem}
\usetikzlibrary{patterns}
\usetikzlibrary{decorations.text}
\begin{document}

\setcounter{chapter}{11}
\chapter{Acides--bases et précipitation}

\section{Réactions acide-base}
\subsection{Définitions}%
\label{sub:definitions_acidebase}
\begin{definition}
  Un acide est une espèce chimique capable de céder un proton \ce{H+} en donnant naissance à sa base conjuguée selon la réaction
  \begin{equation}
    \text{Acide} \ce{->} \text{Base} + \ce{H+}
  \end{equation}
\end{definition}

Les  espèces pouvant être à la fois donneur et accepteur de proton sont appelées espèces amphotères ou ampholytes. Par exemple l'eau (\ce{H2O}) est une espèce amphotère, elle peut accepter ou céder un proton
\begin{align}
  & \ce{H2O + H+ -> H3O+} \quad \text{l'eau est une base}\\
  & \ce{H2O -> HO- + H+} \quad \text{l'eau est un acide}
\end{align}

\subsection{Constante d'acidité}%
\label{sub:constante_d_acidite}

Lors de la mise en solution aqueuse d'un acide \ce{A}, il réagit avec l'eau qui se comporte comme une base. On a alors les deux couples acide--base suivants 
\begin{align}
  \ce{A(acide)} &\ce{<=>} \ce{B(base) + H+}\\
  \ce{H2O(base) + H+} &\ce{<=>} \ce{H3O+ (acide)}\\
\end{align}
ce qui conduit à la réaction suivante :
\begin{equation}
  \ce{A + H2O  <=> B + H3O+}
\end{equation}
Cette réaction atteint un état d'équilibre chimique dont la constante d'équilibre $K_a$ est appelée \emph{constante d'acidité} de \ce{A}. 
\begin{eqencadre}
  K_a = \frac{[\ce{H3O+}][\ce{B}]}{c^\circ[\ce{A}]} 
\end{eqencadre}
\newcommand{\pKa}{\mathrm{p}K_a}
on définit également $\pKa = -\log(K_a)$ 

Lorsque l'on met une base \ce{B} en solution aqueuse, il se produit les demi-réactions acide--base 
\begin{align}
  \ce{H+ + B} &\ce{<=> A}\\
  \ce{H2O} &\ce{<=> HO- + H+}
\end{align}
ce qui conduit à l'équation de réaction
\begin{equation}
  \ce{B + H2O <=> A + HO-}
\end{equation}

Remarque : l'eau est une espèce amphotère, elle donne lieu à la réaction acide-base
\begin{equation}
  \ce{2H2O <=> H3O+ + HO-}
\end{equation}
dont la constante de réaction est, à \SI{25}{\celsius} :  
\begin{equation}
  K_e = \frac{1}{\cz^2}[\ce{H3O+}][\ce{HO-}] = \num{e-14}
  \label{eq:Ke}
\end{equation}
soit $\mathrm{p}Ke = 14$. La relation~\eqref{eq:Ke} sera toujours vraie en solution aqueuse, mais la valeur de $K_e$ dépend de la température. 

\begin{definition}
  On appelle \textbf{acide fort}, un acide dont la réaction avec l'eau est totale, l'acide ne peut pas exister dans l'eau, il n'y a pas de constante d'acidité.

  On appelle \textbf{base forte}, une base dont la réaction avec l'eau est totale. 

  Les acides et bases dont la réaction avec l'eau atteint un état d'équilibre sont qualifiés de \textbf{faibles}.
\end{definition}

\subsection{Domaines de prédominance}%
\label{sub:domaines_de_predominance}

\newcommand{\pH}{\mathrm{pH}}
\begin{definition}
  On définit le $\pH$ d'une solution aqueuse par
  \begin{equation}
    \pH = -\log(a_{\ce{H3O+}}) = - \log \left( \frac{[\ce{H3O+}]}{c^\circ} \right) 
  \end{equation}
\end{definition}
La constante d'acidité d'un couple acide/base \ce{A}/\ce{B} est:
\begin{equation}
  K_a = \frac{[\ce{H3O+][\ce{B}]}}{c^\circ[\ce{A}]} \Leftrightarrow \pKa = -\log\left( \frac{[\ce{H3O+}]}{c^\circ}  \right) - \log\left( \frac{[\ce{B}]}{[\ce{A}]} \right) = \pH - \log\left( \frac{[\ce{B}]}{[\ce{A}]} \right) 
\end{equation}
On obtient alors la relation
\begin{eqencadre}
 \pH = \pKa + \log\left( \frac{[\ce{B}]}{[\ce{A}]}\right) 
\end{eqencadre}
On a alors :
\begin{itemize}
  \item lorsque $\pH<\pKa$, $[\ce{A}]>[\ce{B}]$, l'espèce acide \ce{A} est prédominante ;  
  \item lorsque $\pH>\pKa$, $[\ce{B}]>[\ce{A}]$, l'espèce basique \ce{B} est prédominante ;
  \item lorsque $\pH = \pKa$, $[\ce{A}] = [\ce{B}]$.   
\end{itemize}
On trace le \textbf{diagramme de prédominance} :
\begin{center}
  \begin{tikzpicture}
    \draw[-stealth] (0,0) -- (10, 0) node[right]{$\pH$};
    \draw[thick] (5, 1) -- (5, -0.2) node[below]{$\pKa$};
    \draw (2.5, 0) node[above, align=center]{A prédomine\\$[\ce{A}]>[\ce{B}]$};
    \draw (7.5, 0) node[above, align=center]{B prédomine\\$[\ce{B}]>[\ce{A}]$};
  \end{tikzpicture}
  \captionof{figure}{Diagramme de prédominance des deux espèces A et B d'un couple acide/base}
\end{center}
%
Lorsqu'on trace les proportions de A ou B en fonction du $\pH$, on obtient une \textbf{courbe de distribution :}

\begin{minipage}{\textwidth}
\begin{center}
  \begin{tikzpicture}
  \begin{axis}[
  xmin=0, xmax=14,
  ymin=0,
  ylabel=Proportion de l'espèce (\si{\percent}),
  xlabel=$\pH$, 
  extra x ticks = {5},
  extra x tick labels = {$\pKa$},
  ]
    \addplot[no marks,domain=0:14, thick, smooth] {100*10^(x-5)/(1+10^(x-5))} node[pos=0.98, fill=white]{B};
    \addplot[no marks,domain=0:14, thick, smooth, dashed] {100/(1+10^(x-5))} node[pos=0.02, fill=white]{A};
    \draw[dashed] (axis cs:5,0) -- (axis cs:5, 120); 
    \draw[dashed] (axis cs:0,50) -- (axis cs:14, 50); 
  \end{axis}
  \end{tikzpicture}

  \captionof{figure}{Courbes de distribution des espèces acide (A) et basique (B) d'un couple acide/base en fonction du $\pH$.} 
\end{center}
\end{minipage}
Les diagrammes de prédominance ou courbes de distribution permettent de déterminer quelles sont les espèces majoritaires présentes en solution à un $\pH$ donné.

On peut aussi donner le diagramme de prédominance pour un \textbf{polyacide} (acide qui peut céder plusieurs protons). Par exemple \ce{H3PO4}, donne lieux aux équilibres suivants :
\begin{align}
  &\ce{H3PO4 + H2O <=> H2PO4- + H3O+} \quad  &{\pKa}_1 &= \num{2.15} \\
  &\ce{H2PO4- + H2O <=> HPO4^{2-} + H3O+} \quad & {\pKa}_2 &= \num{7.2}\\
  &\ce{HPO4^{2-} + H2O <=> PO4^{3-} + H3O+} \quad & {\pKa}_3 &= \num{12.42}
\end{align}
On a le diagramme de prédominance suivant :
\begin{center}
  \begin{tikzpicture}
    \draw[-stealth] (0,0) -- (12, 0) node[right]{$\pH$};
    \draw[thick] (3, 1) -- (3, -0.2) node[below]{${\pKa}_1$};
    \draw[thick] (6, 1) -- (6, -0.2) node[below]{${\pKa}_2$};
    \draw[thick] (9, 1) -- (9, -0.2) node[below]{${\pKa}_3$};
    \draw (1.5, 0) node[above, align=center]{\ce{H3PO4}};
    \draw (4.5, 0) node[above, align=center]{\ce{H2PO4-}};
    \draw (7.5, 0) node[above, align=center]{\ce{HPO4^{2-}}};
    \draw (10.5, 0) node[above, align=center]{\ce{PO4^{3-}}};
  \end{tikzpicture}
  \captionof{figure}{Diagramme de prédominance des espèces du polyacide \ce{H3PO4}.}
\end{center}
Lorsque le $\pH$ d'une solution se trouve \emph{assez loin} (plus de 1 de différence de $\pH$) d'une frontière du diagramme de prédominance, on pourra considérer que seule l'espèce prédominante est présente en solution.
\subsection{Réactions acide--base}%
\label{sub:combinaisons_de_reactions}
Lorsqu'on met en solutions un acide et une base appartenant à deux couples différents, ils peuvent réagir pour donner la base et l'acide conjugués de chaque couple. Par exemple considérons les deux couples suivants :
\begin{center}
\begin{tabular}{lll}
\toprule  
Acide & Base & $\pKa$ \\
\midrule
acide acétique \ce{CH3COOH} & ion acétate \ce{CH3COO-} & \num{4.75} \\
acide borique \ce{H3BO3} & ion borate \ce{H2BO3-} & \num{9.20}\\
\bottomrule
\end{tabular}
\end{center}

Les deux équations de réaction avec l'eau sont :
\begin{align}
  &\ce{CH3COOH + H2O <=> CH3COO- + H3O+} & \quad & {K_a}_1 = \frac{[\ce{CH3COO-}][\ce{H3O+}]}{c^\circ[\ce{CH3COOH}]} \label{eq:ab1}\\
  &\ce{H3BO3 + H2O <=> H2BO3- + H3O+}& \quad  & {K_a}_2 = \frac{[\ce{H2BO3-}][\ce{H3O+}]}{c^\circ[\ce{H3BO3}]}\label{eq:ab2}
\end{align}

Si on met en solution de l'acide acétique (\ce{CH3COOH}) et l'ion borate (\ce{H2BO3-}), il peut se produire la réaction :
\begin{equation}
  \ce{CH3COOH + H2BO3- <=> CH3COO- + H3BO3}\label{eq:abtot}
\end{equation}
dont la constante d'équilibre s'écrit :
\begin{equation}
  K = \frac{[\ce{CH3COO-}] [\ce{H3BO3}]}{[\ce{CH3COOH}][\ce{H2BO3-}]} = \frac{{K_a}_1}{{K_a}_2} = \frac{10^{-4,75}}{10^{-9,20}} = 10^{4,45}
\end{equation}
Formellement, on a \eqref{eq:abtot}=\eqref{eq:ab1}-\eqref{eq:ab2} et donc la constante d'équilibre de la réaction~\eqref{eq:abtot} est égale à la constante d'équilibre de l'équation~\eqref{eq:ab1} divisée par la constante d'équilibre de l'équation~\eqref{eq:ab2}.

Lorsque plusieurs espèces acido-basiques sont introduites en solution, il peut se produire plusieurs réactions simultanément et le problème est difficile à résoudre exactement. En pratique il y a souvent une réaction qui se produit beaucoup plus que les autres et qui déterminer quasiment totalement l'état d'équilibre final, il s'agit de la \textbf{réaction prépondérante}. 

La réaction prépondérante est celle qui a lieu entre l'acide le plus fort ($\pKa$ le plus faible) et la base la plus forte ($\pKa$  le plus élevé) parmi les espèces présentes en solution. Pour déterminer la réaction prépondérante, on dispose toutes les espèces présentes en solution sur une échelle de $\pKa$. 

Par exemple, considérons une solution aqueuse dans laquelle on mélange de l'éthanoate de sodium (\ce{CH3COO- + Na+}), de l'acide chlorhydrique (\ce{H3O+ + Cl-}) et du nitrate d'amonium (\ce{NH4+ + NO3-}). On place les différents couples acide/base mis en jeux sur une échelle de $\pKa$ en indiquant les espèces majoritaires (introduites en solution)
\begin{center}
  \begin{tikzpicture}
    \draw[-stealth] (0,0) -- (10, 0) node[right]{$\pKa$};
    \def\h{0.1}
    \draw (1,-\h) node[below, yshift=-0.3cm, draw, inner sep=2pt]{\ce{H2O}} -- ++(0, 2*\h) node[above]{0} node[above, yshift=0.8cm, draw, inner sep=2pt]{\ce{H3O+}};
    \draw (3,-\h) node[below, yshift=-0.3cm, draw, inner sep=2pt]{\ce{CH3COO-}} -- ++(0, 2*\h)  node[above]{\num{4.8}} node[above, yshift=0.8cm]{\ce{CH3COOH}};
    \draw (5,-\h) node[below, yshift=-0.3cm]{\ce{NH3}} -- ++(0, 2*\h) node[above]{\num{9.2}} node[above, yshift=0.8cm, draw, inner sep=2pt]{\ce{NH4+}};
    \draw (7,-\h) node[below, yshift=-0.3cm]{\ce{HO-}} -- ++(0, 2*\h) node[above]{\num{14}} node[above, yshift=0.8cm, draw, inner sep=2pt]{\ce{H2O}};
  \end{tikzpicture}
\end{center}

Dans ce cas, l'acide le plus fort est \ce{H3O+} et la base la plus forte est \ce{CH3COO-}, la réaction prépondérante sera donc
\begin{equation}
  \ce{H3O+ + CH3COO- <=> H2O + CH3COOH}
\end{equation}
Les autres réactions possibles ne modifieront que marginalement l'état d'équilibre final.
\subsection{Quelques espèces à connaître}%
\label{sub:quelques_couples_a_connaitre}
On donne dans le tableau ci-dessous quelques acides et bases dont il faut connaître les noms, formules et s'il sont faibles ou forts.

\begin{center}
  \begin{tabular}{@{}lll@{}}
    \toprule
    Espèce &  Nature \\
    \midrule
    Acide sulfurique (\ce{H2SO4}) & Acide fort \\
    Acide chlorhydrique (\ce{HCl})  & Acide fort \\
    Acide nitrique (\ce{HNO3})  & Acide fort \\
    Acide phosphorique (\ce{H3PO4})  & Acide faible ($\pKa=\num{2.15}$)  \\
    Acide acétique (\ce{CH3COOH}) & Acide faible ($\pKa=\num{4.76}$)  \\
    Soude (hydroxyde de sodium) (\ce{NaOH}) & Base forte  \\
    Ion hydrogénocarbonate (\ce{HCO3-}) & Acide faible ($\pKa=\num{10.32}$ ) et base faible ($\pKa=\num{6.37}$ )  \\
    Ammoniac (\ce{NH3}) & Base faible ($\pKa=\num{9.23}$)  \\
    \bottomrule
  \end{tabular}
\end{center}

\section{Dissolution et précipitation}%
\label{sec:precipitation}

\subsection{Définitions}%
\label{sub:definitions}

\begin{definition}
  Une \textbf{dissolution} est une réaction au cours de laquelle, un solide ionique est incorporé à un solvant pour former une phase homogène. Les ions constituant le solide sont séparés au cours de la dissolution.
  \begin{equation}
    \ce{CA(s) <=> C+(aq) + A-(aq) }
    \label{eq:dissolution}
  \end{equation}
\end{definition}
La constante d'équilibre $K_S$ de la réaction de dissolution s'appelle le \textbf{produit de solubilité}. Dans le cas de l'équation~\eqref{eq:dissolution}, le produit de solubilité s'écrit 
\begin{equation}
  K_S = \frac{1}{{c^\circ}^2}[\ce{C+}][\ce{A-}]
\end{equation}
On donne aussi souvent $pK_S = -\log{K_S}$. En général, le produit de solubilité augmente avec la température. 

\begin{definition}
 Une \textbf{précipitation} est une réaction au cours de laquelle des espèces dissoutes réagissent pour former une espèce solide.  
  \begin{equation}
    \ce{ C+(aq) + A-(aq) <=> CA(s) }
    \label{eq:precipitation}
  \end{equation}
\end{definition}
C'est la réaction inverse de la dissolution. Sa constante d'équilibre est 
\begin{equation}
  K=\frac{{c^\circ}^2}{[\ce{C+}][\ce{A-}]} = \frac{1}{K_S}
\end{equation}
où $K_S$ est le produit de solubilité de l'équation de dissolution correspondante.

\subsection{Condition de précipitation}%
\label{sub:condition_de_precipitation}
Considérons la réaction de dissolution du chlorure d'argent
\begin{equation}
  \ce{AgCl(s) <=> Ag+ + Cl-}
\end{equation}
de produit de solubilité $K_S$. Si on introduit dans l'eau uniquement des ions \ce{Ag+} et \ce{Cl-}, le quotient réactionnel s'écrit 
\begin{equation}
  Q = \frac{1}{{c^\circ}^2}[\ce{Ag+}][\ce{Cl-}]
\end{equation}

\begin{itemize}
  \item Si $Q<K_S$, la réaction doit se produire dans le sens direct (\ce{->}), or il n'y a pas de \ce{AgCl(s)} en solution donc la réaction ne peut pas se produire, il ne se passe rien.
  \item Si $Q>K_S$, la réaction se produit dans le sens indirect (\ce{<-}) et il va y avoir précipitation de \ce{AgCl(s)}.
\end{itemize}

Lorsque la réaction atteint un état d'équilibre entre l'espèce solide et les espèces dissoutes, on a $Q=K_S$, et on dit que la solution est \textbf{saturée} 

\begin{application}
  Le produit de solubilité du sel de cuisine (\ce{NaCl}) est $K_S=39$. Lorsqu'on introduit $\ce{NaCl(s)}$ dans l'eau, on a 
  \begin{center}
    \begin{tabular}{@{}lllll@{}}
\toprule
\ce{NaCl(s)} & \ce{<=>} & \ce{Na+(aq)} & \ce{+} & \ce{Cl-(aq)} \\
\midrule
$n_0$ & & $0$ & & $0$ \\
$n_0-\xi$ & & $\xi$ & & $\xi$ \\
\bottomrule
    \end{tabular}
  \end{center}
  À l'équilibre, on a $\left( \frac{\xi}{c^\circ V} \right)^2 = K_S$. Et donc $\frac{\xi}{V} = c^\circ \sqrt{K_S} \approx{ \luaexec{SI(math.sqrt(39), 1, "\\mol\\per\\litre")}}$, ce qui correspond à la concentration de \ce{NaCl} dissoute à saturation. La masse molaire de \ce{NaCl} étant de $M=\SI{58.5}{\gram\per\mole}$, à saturation il y a $S=\luaexec{SI(math.sqrt(39)*58.5, 1, "\\gram\\per\\litre")}$ de sel dissout dans l'eau. $S$ est la \textbf{solubilité} de \ce{NaCl}.
\end{application}


\newcommand{\pKs}{\mathrm{p}K_S}
\subsection{Domaine d'existence}%
\label{sub:domaine_d_existence}
On considère la réaction de dissolution 
\begin{equation}
  \ce{CA(s) <=> C+(aq) + A-(aq)}
\end{equation}
On ajoute dans une solution contenant uniquement des ions \ce{C+} (un anion n'intervenant pas dans la réaction) à la concentration $c_i$ une solution concentrée contenant des ions \ce{A-(aq)}. Un précipité apparait lorsque la concentration de \ce{A-} devient supérieure à une concentration limite $[\ce{A-}]_\text{lim}$ telle que : 
\begin{equation}
  \frac{1}{\cz^2}[\ce{A-}]_\text{lim}\underbrace{[\ce{C+}]}_{c_i} = K_S \Leftrightarrow
  \log \left( \frac{[\ce{A-}]_\text{lim}}{\cz} \right)  + \log\left( \frac{c_i}{\cz} \right)  = \log(K_S) \Leftrightarrow
  \mathrm{p}\ce{A}_\text{lim} = \pKs + \log\left( \frac{c_i}{\cz} \right) 
\end{equation}
%
On peut alors tracer le diagramme d'existence du précipité \ce{CA(s)} :
\begin{center}
  \begin{tikzpicture}
    \draw[-stealth] (0,0) -- (10, 0) node[right]{$\mathrm{pA}$};
    \draw[thick] (5, 1) -- (5, -0.2) node[below]{$\mathrm{pA}_\text{lim}$};
    \draw (2.5, 0) node[above, align=center]{Existence de\\$\ce{CA(s)}$};
    \draw (7.5, 0) node[above, align=center]{Absence de\\$\ce{CA(s)}$};
  \end{tikzpicture}
  \captionof{figure}{Diagramme d'existence d'un précipité dans une solution donnée.}
\end{center}

Remarque : la valeur limite de $\mathrm{p}A$ dépend de la solution considérée, contrairement aux diagrammes de prédominance d'un couple acide/base où la position de la frontière dépend uniquement du couple acide/base.

\subsection{Facteurs influençant la solubilité}%
\label{sub:facteurs_influencant_la_solubilite}
La solubilité d'une espèce solide dans une solution aqueuse peut dépendre de plusieurs facteurs :
\begin{itemize}
  \item En générale, la solubilité d'une espèce augmente avec la température, c'est à dire que le produit de solubilité augmente avec la température.

  \item Si la solution contient initialement des ions qui composent une espèce solide, la solubilité du solide en question est réduite. Par exemple la solubilité de \ce{NaCl(s)} sera moins importante dans une solution de soude (\ce{Na+ + HO-}) que dans une solution d'hydroxyde de potassium (\ce{K+ + HO-}).

  \item Si l'un des ions constituant l'espèce solide a des propriétés acido-basiques, la solubilité de l'espèce dépend du $\pH$ de la solution. 
\end{itemize}




\end{document}
